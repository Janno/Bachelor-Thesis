\chapter*{Abstract}
\label{chap:abstract}
%Existing formalizations of regular languages in constructive settings are mostly limited to regular expressions and finite automata. 
%Furthermore, these usually require in the order of 10,000 lines of code.\todo{Citations?} %
%The goal of this thesis is to show that an extensive, yet elegant formalization of regular languages can be achieved in constructive type theory. 
%In addition to regular expressions and finite automata, our formalization includes the Myhill-Nerode theorem. 
%The entire development weighs in at approximately 3,300 lines of code.\todo{Reduce \& update} %


We give a constructive formalization of the equivalence between regular expressions, finite automata and the Myhill-Nerode characterization. 
We give procedures to convert between these characterizations and prove their correctness.
Our development is done in the proof assistant \coq.
We make use of the \ssreflect\ plugin which provides support for finite types and other useful infrastructure for our purpose.
Our goal was to make the formalization as concise as possible.
The entire development consists of approximately 2700 lines of code.
