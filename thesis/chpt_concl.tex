\chapter{Conclusion}
We give a short overview of the theorems presented in this thesis and their corresponding statements in the \coq\ development.
We then evaluate our choice of the \ssreflect\ plugin.
Finally, we discuss opportunities for future work. 

\section{Results}
\label{chap:conclusion}
Theorem \ref{dfa_to_nfa} and Theorem \ref{nfa_to_dfa} 
show that deterministic and non-deterministic finite automata are equally expressive.
For this, we construct two functions \lstinline{dfa_to_nfa} and \lstinline{nfa_to_dfa} to convert between the two characterizations.
\fcode{}{}{automata_dfa_to_nfa_correct_head}
\fcodeblock{}{}{automata_nfa_to_dfa_correct_head}

We show in Theorem \ref{re_to_dfa} that there is an equivalent DFA for every extended regular expression. 
For this, we construct a function \lstinline{re_to_dfa} to compute an equivalent DFA from an extended regular expressions.
%This is by far the most extensive proof of the entire development.
\fcode{}{}{re_fa_re_to_dfa_correct_head}

Building on that, we prove the decidability of equivalence of regular expressions in Theorem \ref{re_equiv_correct} with the help of a decision procedure for equivalence of finite automata.
We give a function \lstinline{re_equiv} to decide the equivalence of regular expressions.
\fcode{}{}{re_fa_re_equiv_correct_head}

Theorem \ref{dfa_to_re} shows that we can give an equivalent regular expression for every automaton. 
We construct a function \lstinline{dfa_to_re} to compute the regular expression.
\fcode{}{}{transitive_closure_dfa_to_re_correct_head}

Based on this and the results from previous chapters, we also show that extended and standard regular expressions are equally expressive and, thus, that extended regular expressions and regular languages are equally expressive.
We give a function \lstinline{ext_re_to_std_re} which constructs an equivalent standard regular expression for every extended regular expression.
%The proof of this theorem is quite algorithmic. 
%Still, it is small in size.
\fcode{}{}{transitive_closure_ext_re_to_std_re_standard_head}
\fcodeblock{}{}{transitive_closure_ext_re_to_std_re_correct_head}

With Theorem \ref{dfa_to_myhill}, we prove that we can construct a Myhill relation for a language from a DFA for that language.
\fcode{}{}{myhill_nerode_dfa_to_myhill_head}.

We prove in Theorem \ref{myhill_to_weak_nerode} that every Myhill relations is also a weak Nerode relation.
\fcode{}{}{myhill_nerode_myhill_to_weak_nerode_head}

Theorem \ref{weak_nerode_to_nerode} shows that, if there is a weak Nerode relation, the Nerode relation is of finite index.
\fcode{}{}{myhill_nerode_weak_nerode_to_nerode_head}

Finally, we prove in Theorem \ref{nerode_to_dfa} that, if we are given a Nerode relation of finite index for a language, we can construct a DFA that accepts this language.
\fcode{}{}{myhill_nerode_nerode_to_dfa_correct_head}

\section{\ssreflect}
We make extensive use of \ssreflect's features in our development.
The formalization of finite automata depends crucially on finite types (and, to a lesser extend, finite sets).

We also employ the \lstinline{reflect} paradigm whenever possible.
It offers a very convenient way of working with propositional and boolean predicates at the same time.
The built-in support for changing from propositional to boolean statements lets us choose the most appropriate representation for the task at hand.


Furthermore, the very extensive library of general purpose lemmas in \ssreflect\ enables us to focus on high-level proofs. 
The sole exception to this is the \lstinline{allbutlast} predicate we need for Theorem \ref{dfa_to_re}.
However, even in this case, we can mostly rely on the lemmas for \lstinline{all} provided by \ssreflect.
All we need to do is provide a thin layer between the two predicates.

Additionally, the scripting language offered by \ssreflect\ leads to very concise proof scripts. 
It succeeds in removing some of the bookkeeping overhead. % that is usually present in this kind of proofs.

There are several disadvantages to \ssreflect. 
One is that it not as widely used as \coq\ itself. 
This means that the group of people who can understand the 
proof scripts is small.
However, in some cases, it might be sufficient to explain a small subset of
\ssreflect\ in order to give an understandable presentation of the formalized statements. 

We also lose practical executability. 
Specifically, the implementation underlying finite types does not lend itself to computation. 
Since practical executability is not always a requirement, this restriction may not be relevant to some projects, as is the case with our development.

Based on this considerations, we believe that the use of \ssreflect\ is very beneficial to formalizations that do not require executability, especially if there is algorithmic content. 

%In our formalization, we only needed a small subset of \ssreflect's library. 
%The interested reader can find the full extent of the project online%
%\footnote{\url{http://coqfinitgroup.gforge.inria.fr/}}.


\section{Future Work}
There a several possible extensions to our development.
Additionally, there are some topics that are not so much extensions but rather candidates for future formalizations.
%There is also a lot of potential in reducing the size of the development. 


\subsection[$\varepsilon$-Transitions]{\texorpdfstring{$\varepsilon$-Transitions}{Epsilon-Transitions}}
We have avoided $\varepsilon$-transitions in our formalization. 
Non-deterministic finite automata with $\varepsilon$-transitions and regular expressions are equally expressive.
They are, as we have shown, unnecessary to derive the results proven in this thesis.
Nonetheless, it would be very interesting to add them to the list of formalized characterizations of regular languages.

\subsection{Pumping Lemma}
The Pumping Lemma \cite{BarHillelPerlesShamir61Formal} gives a sufficient condition for the non-regularity of a language. 
It is a well-known part of the theory of regular languages and, thus, a good candidate for an extension to our development.

\subsection{Regular Grammars}
Another characterization of regular languages is given by regular grammars.
Regular grammars seem to enjoy less popularity than other characterizations.
A formalization of formal grammars in general would also be a good starting point to formalize other parts of the
Chomsky Hierarchy \cite{Chomsky1956}. 
The context-free languages could be a good candidate for a formalization.
We speculate that pushdown automata could be formalized similarly to how we formalized finite automata.

\subsection{$\omega$-regular languages}
A possible next step after the formalization of regular languages is a formalization of $\omega$-regular languages.
There does not seem much literature on formalizing this topic.
Such a development could include all commonly used acceptance criteria on $\omega$-automata.
This would also make for a good opportunity to study \coq's co-inductive definitions. 
