\chapter{\coq\ and \ssreflect}

\paragraph{}
We decided to employ the Small Scale Reflection Extension (\textbf{\ssreflect}) for the \textbf{\coq} proof assistant. 
The most important factors in this decision were \ssreflect's excellent support for finite types, list operations and graphs. \ssreflect{} also introduces an alternative scripting language that can often be used to shorten the bookkeeping overhead of proofs considerably.

\section{\coq}

\section{\ssreflect}

\subsection{Finite Types and Ordinals}
The most important feature of \ssreflect\ for our purpose are finite types. 
\ssreflect\ provides boolean versions of the universal and existential quantifiers on finite types, \lstinline{forallb} and \lstinline{existsb}.
We can compute the number of elements in a finite type F with \lstinline{#|F|}.
\lstinline{enum} gives a list of all items of a finite type. 
Finite types also come with enumeration functions which provide a consistent ordering. 
The corresponding functions are \lstinline{enum_rank} and \lstinline{enum_val}. 
The input of \lstinline{enum_val} and the result of \lstinline{enum_rank} are ordinals, i.e. values in \lstinline{[0, \#|F|-1]}. 
The corresponding type can be written as \lstinline{I_#|F|}

\subsection{Boolean Reflection}
\ssreflect\ offers boolean reflections for decidable propostions. 
This allows us to switch back and forth between equivalent boolean and propositional predicates.

\subsection{Boolean Predicates}
We make use of \ssreflect's syntax to specify boolean predicates.
This allows us to specify predicates in a way that resembles set-theoretic notation, e.g. 
\lstinline{[pred x | <boolean expression in x> ]}.
Furthermore, we can use the functions \lstinline{pred1} and \lstinline{pred0} to specify the singleton predicate and the empty predicate, respectively.
The complement of a predicate can be written as \lstinline{[predC p]}. 
The syntax for combining predicates is \lstinline{[pred? p1 & p2 ]}, with \lstinline{?} being replaced with one of 
\lstinline{U} (union), \lstinline{I} (intersection) or \lstinline{D} (difference).
There is also syntax for the preimage of a predicate under a function
which can be written as \lstinline{[preim f of p]}.

%The resulting predicates are called collective predicates. 
%\ssreflect\ allows us to use infix notation, i.e. $x \in p$, for these predicates.

\paragraph{}
There are also applicative (functional) versions of of \lstinline{predC}, \lstinline{predU}, \lstinline{predI}, \lstinline{predD} which are functions that take predicates and return predicates.

