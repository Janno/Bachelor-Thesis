
\chapter{Myhill-Nerode}
\label{chap:MN}

\paragraph{}

The last characterization of regular languages that we consider is given by the Myhill-Nerode theorem.
\todo{What to write here?}

\section{Definition}

The following definitions (roughly following \cite{DBLP:books/daglib/0088160}) will lead us to the statement of the Myhill-Nerode theorem.
%Let $\equiv$ be an equivalence relation on $\Sigma^*$. Let $L$ be a language over $\Sigma$.

\begin{definition}
    The \textbf{equivalence class} of $u \in \Sigma^*$ w.r.t. $\equiv$ is the set of all $v$ such that $u \equiv v$.
    It is denoted by $[u]_\equiv$.
\end{definition}

\begin{definition}
    $\equiv$ is of \textbf{finite index} if and only if the set of $\{[u]_\equiv \; | \; u \in \Sigma^* \}$ is finite.
\end{definition}

\paragraph{}
Due to the lack of native support for quotient types in \coq, 
we formalize equivalence relations of finite index 
as functions from $\Sigma^*$ to a finite type $X$.

\begin{definition}
    \label{equiv_f}
    Let $f: \Sigma^* \mapsto X$ be such a function.
    The relation $\equiv_f$ is defined as
    \begin{equation*}
        \{ (u, v) \; | \; u,v \in \Sigma^* \wedge f(u) = f(v) \}.
    \end{equation*}
    For all $w \in \Sigma^*$, $f(w)$ can be seen as an equivalence class of $\equiv_f$.
\end{definition}

\paragraph{}
It is easy to see that $\equiv_f$ is an equivalence relation. 
Furthermore, from the finiteness of $F$, it follows that $\equiv_f$ is of finite index.
\todo{Lemmas for this?}%

\begin{definition}
    Let $f$ be as above. 
    Let $x \in X$. $w \in \Sigma^*$ is a \textbf{representative} of $x$ if and only if $f(w) = x$.
    We write $\crep{x}$ to denote any representative of $x$.
\end{definition}



\paragraph{}
Our formalization of equivalence relations of finite support requires the function $f$ to be surjective. 
Mathematically, this is not a restriction since empty equivalence classes can be disregarded.
In \coq, however, it is required in order to be able to give a representative of every equivalence class.


\code{}{}{myhill_nerode_Fin_Eq_Cls}

\subsubsection{Myhill Relations}

\begin{definition} Let $\equiv$ be an equivalence relation. 

\begin{enumerate}[label=(\roman*)]

    \item\label{right_congruent}
                %\begin{definition}
        $\equiv$ is \textbf{right congruent} if and only if for all $u, v \in \Sigma^*$ and $a \in \Sigma$,
        \begin{equation*}
            u \equiv v \Rightarrow
            u \cdot a \equiv v \cdot a.
        \end{equation*}
                %\end{definition}


    \item\label{refinement}
                %\begin{definition}
        $\equiv$ \textbf{refines} $L$ if and only if for all $u,v \in \Sigma^*$,
        \begin{equation*}
            u \equiv v \Rightarrow
            (u \in L \iff v \in L).
        \end{equation*}
                %\end{definition}

    \item\label{finite_index}
                %\begin{definition}
        $\equiv$ is of \textbf{finite index} if and only if it has finitely many equivalence classes, i.e.
        \begin{equation*}
            \{[u]_\equiv \; | \; u \in \Sigma^*\} \mbox{ is finite }
        \end{equation*}
                %\end{definition}

\end{enumerate}
\end{definition}

\begin{definition}
    \label{Myhill_Rel}
    An equivalence relation is a \textbf{Myhill relation}%
    \footnote{Myhill relations are commonly referred to as ``Myhill-Nerode relations''. 
    In this thesis, it makes sense to split the concept of a Myhill relation from that of Nerode relation.}
    if and only if it satisfies \ref{right_congruent}, \ref{refinement} and \ref{finite_index} \cite{DBLP:books/daglib/0088160}.
\end{definition}

\paragraph{}
Building on our formalization of equivalence relations of finite support, 
we only need to give formalizations of \ref{right_congruent} and \ref{refinement}.

\code{}{}{myhill_nerode_right_congruent}
\codeblock{}{}{myhill_nerode_refining}
\codeblock{}{}{myhill_nerode_Myhill_Rel}

\paragraph{}
Myhill relations correspond to the equivalence relations 
defined as the pairs of words $(u, v)$ whose runs on a DFA $A$ end in the same state. 
These relations are right congruent, refine $\lang{A}$ and of finite index as $A$ has finitely many states. 
We will later give a formal proof of this.

\subsubsection{Nerode Relations}

\begin{definition}
    \label{equal_suffix}
    Let $u, v \in \Sigma^*$. We say that $u$ and $v$ are \textbf{invariant under concatenation} w.r.t. L if and only if
    \begin{equation*}
        \forall w \in \Sigma^*. \; uw \in L \Leftrightarrow vw \in L. 
    \end{equation*}
    We write $u \doteq_L v$ when $u$ and $v$ are invariant under concatenation w.r.t L.
\end{definition}

\begin{definition}
    \label{Weak_Nerode_Rel}
    Let $\equiv$ be an equivalence relation. We say that $\equiv$ is a \textbf{weak Nerode relation} if and only if
    \begin{equation*}
        \forall u, v \in \Sigma^*. \; u \equiv v \implies u \doteq_L v.
    \end{equation*}
\end{definition}


\code{}{}{myhill_nerode_equal_suffix}
\codeblock{}{}{myhill_nerode_imply_suffix}
\codeblock{}{}{myhill_nerode_Weak_Nerode_Rel}

\paragraph{}
The notion of a weak Nerode relation is not found in the literature.
We will later prove them weaker than Myhill relations, in the sense that every Myhill relation is also a weak Nerode relation.

\begin{definition}
    \label{Nerode_Rel}
    Let $\equiv$ be an equivalence relation. We say that $\equiv$ is a \textbf{Nerode relation}%
    \footnote{The Nerode relation is sometimes referred to as the ``coarsest Myhill-Nerode relation''.}
    if and only if
    \begin{equation*}
        \forall u, v \in \Sigma^*. \; u \equiv v \iff u \doteq_L v.
    \end{equation*}
\end{definition}

\code{}{}{myhill_nerode_equiv_suffix}
\codeblock{}{}{myhill_nerode_Nerode_Rel}











\section{Minimizing Equivalence Classes}

\paragraph{}
We will prove that weak Nerode relations can be converted into Nerode relations.
For this purpose, we employ the table-filling algorithm to find indistinguishable states under the Myhill-Nerode relation \cite{DBLP:books/daglib/0011126}.
However, we do not rely on an automaton. 
In fact, we use the finite type $X$, i.e., the equivalence classes, instead of states.

\paragraph{}
Given a weak Nerode relation $f$, we construct a fixed-point algorithm.
The algorithm initially outputs the set of equivalence classes that are distinguishable by the inclusion of their class representative in $L$. 
We call the corresponding predicate $\mathit{dist}$ and define it such that
We denote this initial set $\mathit{dist_0}$.
\begin{equation*}
    dist_0 := \{ (x,y)  \in F \times F \, | \, \crep{x} \in L \Leftrightarrow \crep{y} \notin L\}.
\end{equation*}

\code{}{}{myhill_nerode_distinguishable}
\codeblock{}{}{myhill_nerode_distinct0}

\paragraph{}
To find more distinguishable equivalence classes, we have to identify equivalence classes that lead to distinguishable equivalence classes. 
\begin{definition}
    We say that an equivalence class $x$ \textbf{transitions} to $y$ with $a \in \Sigma$ if and only if
    %\begin{equation*}
        $f (\crep{x}\cdot a) = y$
    %\end{equation*}
    We denote $y$ by $\mathit{ext_a}(x)$.
\end{definition}
    
\begin{definition}
    A pair of equivalence classes $(x,y)$ \textbf{transitions} to $(x', y')$ with $a$ if and only if $x$ transitions to $x'$ with $a$ and $y$ transitions to $y'$ with a.
    We denote $(x', y')$ by $\mathit{pext_a}(x,y)$.
\end{definition}

The fixed-point algorithm tries to extend the set of distinguishable equivalence classes by looking for a pair of equivalence classes that transitions to a pair of distinguishable equivalence classes. 
Given a set of of equivalence classes $\mathit{dist}$, we define the set of distinguishable equivalence classes they transition to as
\begin{equation*}
    \mathit{distinct_S}(\mathit{dist}) := \{ (x,y) \; | \; \exists a. \, \mathit{pext_a}(x,y) \in \mathit{dist}\}.
\end{equation*}

\begin{definition}
    \label{unnamed}
    Let $\mathit{dist}$ be a subset of $X \times X$. We define $\mathit{unnamed}$ such that
    \begin{equation*}
        \mathit{unnamed}(\mathit{dist}) := \mathit{dist_0} \cup \mathit{dist} \cup \mathit{distinct_S}(\mathit{dist}).
    \end{equation*}
\end{definition}


\code{}{}{myhill_nerode_ext}

\codeblock{}{}{myhill_nerode_pext}

\codeblock{}{}{myhill_nerode_distinctS}

\codeblock{}{}{myhill_nerode_unnamed}

\begin{lemma}
    \label{dist_monotone}
    $unnamed$ is monotone and has a fixed-point.
\end{lemma}
\begin{proof}
    Monotonicity follows directly from the monotonicity of $\cup$. 
    The number of sets in $F \times F$ is finite. 
    Therefore, $unnamed$ has a fixed point.

\end{proof}
\paragraph{}
Let \textit{\textbf{distinct}} be the fixed point of $unnamed$. 
We write \textit{\textbf{equiv}} for the complement of \textit{distinct} and denote it $\cong$.
We denote \textit{distinct} $\not\cong$.

\begin{lemma}
    \label{equiv_refl}
    $\cong$ is an equivalence relation.
\end{lemma}
\begin{proof}
    It suffices to show that $\mathit{distinct}$ is anti-reflexive, symmetric and ????\todo{a name for this}.
    We do a fixed-point induction.
    \begin{enumerate}
        \item For $\mathit{unnamed}(\mathit{dist}) = \emptyset$ we have anti-reflexivity, symmetry and ???? 
            by the properties of of $\emptyset$.
        \item For $\mathit{unnamed}(\mathit{dist}) = \mathit{dist'}$ 
            we have $\mathit{dist}$ anti-reflexive, symmetric and ????.
            The set of anti-reflexive, antisymmetric and ???? sets is closed under union.
            It remains to show that
            $\mathit{dist_0}$ 
            and $\mathit{distinct_S}(\mathit{dist})$ 
            are anti-reflexive, symmetric and ????.
            
            \todo{too much ????}

            $\mathit{dist_0}$ is anti-reflexive and symmetric by definition. 

            $\mathit{distinct_S}(\mathit{dist})$ can be seen as an intersection 
            of a symmetric subset of $X \times X$ defined by $\mathit{pext_a}$ and the anti-reflexive, symmetric $\mathit{dist}$.
            Thus, $\mathit{distinct_S}(\mathit{dist})$ is anti-reflexive and symmetric.
            
            The set of anti-reflexive and antisymmetric sets is closed under union.
            Therefore, $\mathit{dist'}$ is anti-reflexive and symmetric.
    \end{enumerate}
\end{proof}

\code{}{}{myhill_nerode_equiv_refl_head}
\codeblock{}{}{myhill_nerode_equiv_sym_head}
\codeblock{}{}{myhill_nerode_equiv_trans_head}


\begin{lemma}
    \label{equiv_final}

    Let $u,v \in \Sigma^*$. 
    If $f(u) \cong f(v)$, then $u$ and $v$ are invariant under concatenation, i.e. $f(u) \cong f(v) \implies u \doteq_L v$.
\end{lemma}
\begin{proof}
    Let $w \in \Sigma^*$. We then show the contraposition of the claim: 
    \begin{equation*}
        uw \in L \not\Leftrightarrow vw \in L \implies f(u) \not\cong f(v).
    \end{equation*}    
    We do an induction on $w$ and generalize over $u$ and $v$.
    \begin{enumerate}
        \item For $w = \varepsilon$ we have $u \in L \not\Leftrightarrow v \in L$ which gives us $(f(u),f(v)) \in dist_0$.
            Thus, $f(u) \not\cong f(v)$.
        \item For $w = aw'$ we have $uaw \in L \not\Leftrightarrow vaw \in L$.
            We have to show $f(u) \not\cong f(v)$, i.e. $(f(u), f(v)) \in \mathit{distinct}$.
            By definition of $\mathit{distinct}$, it suffices to show $(f(u), f(v)) \in \mathit{unnamed}(\mathit{distinct})$.

             For this, we prove $(f(u), f(v)) \in \mathit{distinct_S}(\mathit{distinct})$. 
             By $uaw \in L \not\Leftrightarrow vaw \in L$ we have $(f(\crep{u}a), f(\crep{v}a)) \in dist_0$.

             It remains to show that $f(\crep{u}a) \not\cong f(\crep{v}a)$ which we get by inductive hypothesis.
             For this, we need to show that $\crep{u}aw \in L \not\Leftrightarrow \crep{v}aw$.

             By the properties of $f$, we get $\crep{u}aw \in L \Leftrightarrow uaw \in L$ and $\crep{v}aw \in L \Leftrightarrow vaw \in L$.
             Thus, $\crep{u}aw \in L \not\Leftrightarrow \crep{v}aw$.
             
    \end{enumerate}
\end{proof}


\begin{lemma}
    \label{distinct_final}
    Let $u,v \in \Sigma^*$. 
    If $f(u) \not\cong f(v)$, then $u$ and $v$ are \textbf{not} invariant under concatenation, i.e. $f(u) \not\cong f(v) \implies u \not\doteq_L v$.
\end{lemma}
\begin{proof}
    We do a fixed-point induction.
    \begin{enumerate}
        \item For $\mathit{unnamed}(\mathit{dist}) = \emptyset$ we have $(f(u), f(v)) \in \emptyset$ and thus a contradiction. 
        \item For $\mathit{unnamed}(\mathit{dist}) = \mathit{dist'}$ we have $(f(u), f(v)) \in \mathit{dist'}$. 
            We do a case distinction on $\mathit{dist'}$.
            \begin{enumerate}
                \item $(f(u), f(v)) \in \mathit{dist_0}$.
                    We have $u \in L \not\Leftrightarrow v \in L$. 
                    Thus, $u \not\doteq_L v$ as witnessed by $w=\varepsilon$.
                \item $(f(u), f(v)) \in \mathit{dist}$. 
                    By inductive hypothesis, $u \not\doteq_L v$.
                \item $(f(u), f(v)) \in \mathit{distinct_S}(\mathit{dist})$.
                    We have $a \in \Sigma$ with $\mathit{pext_a}(f(u), f(v))) \in \mathit{dist}$.
                    By inductive hypothesis, we get $\crep{u}a \not\doteq_L \crep{v}a$ 
                    as witnessed by $w \in \Sigma^*$ 
                    such that $\crep{u}aw \in L \not\Leftrightarrow \crep{v}aw \in L$.

                    By the properties of $f$, we get $\crep{u}aw \in L \Leftrightarrow uaw \in L$ and $\crep{v}aw \in L \Leftrightarrow vaw \in L$.
                    Thus, we have $u \not\doteq_L v$ as witnessed by $aw$.
            \end{enumerate}
    \end{enumerate}
\end{proof}


\begin{corollary}
    \label{equivP}
    Let $u, v \in \Sigma^*$. We have that
    \begin{equation*}
        f(u) \cong f(v) \iff u \doteq_L v.
    \end{equation*}
\end{corollary}

\code{}{}{myhill_nerode_equiv_equal_suffix_head}
\codeblock{}{}{myhill_nerode_distinct_not_equal_suffix_head}
\codeblock{}{}{myhill_nerode_equivP_head}

\begin{definition}
    \label{f_min}
    Let $w \in \Sigma^*$. We define
    %Then $\mathit{f_{min}}$ is defined such that 
    \begin{equation*}
        \mathit{f_{min}}(w) := \{ x \; | \; x \in X, \; f(w) \cong x \}.
    \end{equation*}
    Note that the domain of $\mathit{f_{min}}$ is finite (since $f$ is finite) and contains no empty sets (due to reflexivity of $\cong$).
\end{definition}

\begin{lemma}
    \label{f_min_surjective}
    $\mathit{f_{min}}$ is surjective.
\end{lemma}
\begin{proof}
    Let $s \in dom(\mathit{f_{min}})$. 
    There exists $x \in X$ such that $x \in s$ since $s \neq \emptyset$. 
    We have $f(x) = f(\crep{x}$) and therefore $f(x) \cong f(\crep{x})$ by reflexivity of $\cong$.
    Thus, $\crep{x}$ is a representative of $s$ since $\mathit{f_{min}}(x) = \mathit{f_{min}}(\crep{x}) = s$.
\end{proof}

\begin{lemma}
    \label{f_minP}
    For all $u,v \in \Sigma^*$ we we have 
    \begin{equation*}
        \mathit{f_{min}}(u) = \mathit{f_{min}}(v) \iff f(u) \cong f(v).
    \end{equation*}
\end{lemma}

\begin{proof}
    ``$\Rightarrow$''
    We have $\mathit{f_{min}}(u) = \mathit{f_{min}}(v)$ and thus $f(u) \cong f(v)$.

    ``$\Leftarrow$''
    We have $f(u) \cong f(v)$. 
    Let $x \in X$.
    It suffices to show that $f(u) \cong x$ if and only if $f(v) \cong x$.
    This follows from symmetry and transitivity of $\cong$.
\end{proof}

\begin{theorem}
    \label{f_min_correct}
    $\mathit{f_{min}}$ is a Nerode relation, i.e. $\mathit{f_{min}}$ is surjective and for all $u,v \in \Sigma^*$ we have
    \begin{equation*}
        \mathit{f_{min}}(u) = \mathit{f_{min}}(v) \iff u \doteq_L v.
    \end{equation*}
\end{theorem}

\begin{proof}
    We have proven surjectivity in lemma \ref{f_min_surjective}. 
    By lemma \ref{f_minP} we have $\mathit{f_{min}}(u) = \mathit{f_{min}}(v)$ if and only if $f(u) \cong f(v)$.
    By corollary \ref{equivP} we have $f(u) \cong f(v)$ if and only if $u \doteq_L v$.
    Thus, $\mathit{f_{min}}(u) = \mathit{f_{min}}(v)$ if and only if $u \doteq_L v$.
\end{proof}

\paragraph{}
The formalization of $\mathit{f_{min}}$ is slightly more involved than the mathematical construction. 
We first need to define the finite type of $\mathit{f_{min}}$'s domain, 
which we do by enumerating all possible values of $\mathit{f_{min}}$.

\code{}{}{myhill_nerode_equiv_repr}
\codeblock{}{}{myhill_nerode_X_min}
\codeblock{}{}{myhill_nerode_f_min}

\paragraph{}
We can then prove lemmas \ref{f_min_surjective}, \ref{f_minP} and theorem \ref{f_min_correct}.

\code{}{}{myhill_nerode_f_min_surjective_head}
\codeblock{}{}{myhill_nerode_f_minP_head}
\codeblock{}{}{myhill_nerode_f_min_correct_head}

\paragraph{}
Finally, we give a function to explicitly convert from the weak Nerode relation $f$ to a Nerode relation.

\code{}{}{myhill_nerode_f_min_fin}
\codeblock{}{}{myhill_nerode_weak_nerode_to_nerode}

\section{Finite Automata and Myhill-Nerode}

\paragraph{}
We prove theorem \ref{MN} by proving it equivalent to the existence of an automaton that accepts $L$.



\subsection{Finite Automata to Myhill-Nerode}


\subsection{Myhill-Nerode to Finite Automata}

\paragraph{}


