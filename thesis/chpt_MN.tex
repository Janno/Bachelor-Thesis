
\chapter{Myhill-Nerode}

\paragraph{}

The last characterization we consider is given by the Myhill-Nerode theorem.

\section{Definition}

\paragraph{}

The following definitions (taken from \cite{DBLP:books/daglib/0088160}) will lead us to the statement of the Myhill-Nerode theorem.
We assume $\equiv$ to be an equivalence relation on $\Sigma^*$, and $L$ to be a language over $\Sigma$.

\begin{enumerate}[label=(\roman*)]

    \item\label{rightcongruent}
                %\begin{definition}
        $\equiv$ is \textbf{right congruent} if and only if for all $x, y \in \Sigma^*$ and $a \in \Sigma$,
        \begin{equation*}
            x \equiv y \Rightarrow
            x \cdot a \equiv y \cdot a.
        \end{equation*}
                %\end{definition}


    \item\label{refinement}
                %\begin{definition}
        $\equiv$ \textbf{refines} $L$ if and only if forall $x,y \in \Sigma^*$,
        \begin{equation*}
            x \equiv y \Rightarrow
            (x \in L \Leftrightarrow y \in L).
        \end{equation*}
                %\end{definition}

    \item\label{finiteindex}
                %\begin{definition}
        $\equiv$ is of \textbf{finite index} if and only if it has finitely many equivalence classes, i.e.
        \begin{equation*}
            \{[x] | x \in \Sigma^*\} \mbox{ is finite }
        \end{equation*}
                %\end{definition}

\end{enumerate}

\begin{definition}
    \label{MN_relation}
    A relation is Myhill-Nerode if and only it satisfies properties \ref{rightcongruent}, \ref{refinement} and \ref{finiteindex}.
\end{definition}

\todo{Fix everything below this line}

\paragraph{} Given a language $L$, the Myhill-Nerode relation $\approx_L$ is defined such that 
\begin{equation*}
    \forall u, v \in \Sigma^*. \,
    u \approx_L v \, \Longleftrightarrow \, 
    \forall w \in \Sigma^*.\, u \cdot w \in L \Leftrightarrow v \cdot w \in L.
\end{equation*}

\code{mn}{Myhill-Nerode relation}{myhill_nerode_MN}

\begin{theorem}{Myhill-Nerode Theorem.}
    \label{MN}
    A language $L$ is regular if and only if $\,\approx_L$ is of finite index.
\end{theorem}

\section{Finite Partitionings and Equivalence Classes}

\paragraph{}
\coq\ does not have quotient types. 
We pair up functions and proofs for certain properties of those functions to emulate quotient types.

\paragraph{} 
A finite partitioning is a function from $\Sigma^*$ to some finite type $F$. 
We use this concept to model equivalent classes in \coq. 
A finite partitioning of the Myhill-Nerode relation is a finite partitioning $f$ that also respects the Myhill-Nerode relation, i.e.,
\begin{equation*}
    \forall u, v \in \Sigma^*. \,
    f(u) = f(v) \Leftrightarrow u \approx_L v.
\end{equation*}


\code{mn_rel}{Finite partitioning of the Myhill-Nerode relation}{myhill_nerode_MN_rel}

\begin{theorem}
    $\approx_L$ is of finite index if and only if there exists a finite partitioning of the Myhill-Nerode relation.
\end{theorem}

\begin{proof}
    If $\approx_L$ is of finite index, we use the set equivalence classes as a finite type and construct $f$ such that
    \begin{equation*}
        \forall w.\, f(w) = [w]_\approx.
    \end{equation*}
    \paragraph{}
    $f$ is a finite partitioning of the Myhill-Nerode relation by definition.

    \paragraph{}
    Conversely, if we have a finite partitioning of the Myhill-Nerode relation, we can easily see that $\approx_L$ must be of finite index since $f$'s values directly correspond to equivalence classes. The image of $f$ is finite. Therefore, $\approx_L$ is of finite index.
\end{proof}

\paragraph{}

A more general concept is that of a refining finite partitioning of the Myhill-Nerode relation:
\begin{equation*}
    \forall u, v \in \Sigma^*. \,
    f(u) = f(v) \Rightarrow u \approx_L v.
\end{equation*}

\code{mn_ref}{Refining finite partitioning of the Myhill-Nerode relation}{myhill_nerode_MN_ref}



\paragraph{}
We require all partitionings to be surjective.
Therefore, every equivalence class $x$ has at least one class representative which we denote $\crep{x}$.
Mathematically, this is not a restriction since there are no empty equivalence classes.
In our constructive setting we would have to give a procedure that builds a minimal finite type $F'$ from $F$ and a corresponding function $f'$ from $\Sigma^*$ to $F'$ such that $f'$ is surjective and extensionally equal to $f$.

\section{Minimizing Equivalence Classes}

\paragraph{}
We will prove that refining finite partitionings can be converted into finite partitionings. 
For this purpose, we employ the table-filling algorithm to find indistinguishable states under the Myhill-Nerode relation (\cite{DBLP:books/daglib/0011126}).
However, we do not rely on an automaton. 
In fact, we use the finite type $F$, i.e., the equivalence classes, instead of states.

\paragraph{}
Given a refining finite partitioning $f$, we construct a fixed-point algorithm.
The algorithm initially outputs the set of equivalence classes that are distinguishable by the inclusion of their class representative in $L$. 
We denote this initial set $dist_0$.
\begin{equation*}
    dist_0 := \{ (x,y)  \in F \times F \, | \, \crep{x} \in L \Leftrightarrow \crep{y} \notin L\}.
\end{equation*}

\paragraph{}
To find more distinguishable equivalence classes, we have to identify equivalence classes that lead to distinguishable equivalence classes. 
\begin{definition}
    We say that a pair of equivalence classes $(x,y)$ \textbf{transitions} to $(x', y')$ with $a$ if and only if
    \begin{equation*}
        f (\crep{x}\cdot a) = x' \wedge f (\crep{y} \cdot a) = y'.
    \end{equation*}
    We denote $(x', y')$ by $ext_a(x,y)$.
\end{definition}

The fixed-point algorithm tries to extend the set of distinguishable equivalence classes by looking for a so-far undistinguishable pair of equivalence classes that transitions to a pair of distinguishable equivalence classes.

\begin{definition}
    \begin{equation*}
        unnamed(dist) := dist_0 \cup dist \cup \{ (x,y) \, | \, \exists a. \, ext_a(x,y) \in dist\}
    \end{equation*}
\end{definition}

\begin{lemma}
    \label{dist_monotone}
    $unnamed$ is monotone and has a fixed-point.
\end{lemma}
\begin{proof}
    Monotonicity follows directly from the monotonicity of $\cup$. 
    The number of sets in $F \times F$ is finite. 
    Therefore, $unnamed$ has a fixed point.

\end{proof}
\paragraph{}
Let \textit{\textbf{distinct}} be the fixed point of $unnamed$.
Let \textit{\textbf{equiv}} be the complement of $distinct$.
\todo{Finish construction}
\begin{theorem}
    \label{MN_MIN}
    $f_min$ is a finite partitioning of the Myhill-Nerode relation on $L$.
\end{theorem}

\todo{Add formalization}



\section{Finite Automata and Myhill-Nerode}

\paragraph{}
We prove theorem \ref{MN} by proving it equivalent to the existence of an automaton that accepts $L$.



\subsection{Finite Automata to Myhill-Nerode}
\paragraph{}
Given DFA $A$, for all words $w$ we define $f(w)$ to be the last state of the run of $w$ on $A$.

\begin{lemma} 
    \label{DFA_MN_F}
    $f$ is a refining finite partitioning of the Myhill-Nerode relation on $\lang{A}$. 
\end{lemma} 
\begin{proof} 
    The set of states of $A$ is finite.
    For all $u, v$ and $w$ we that if $f(u) = f(v) = x$, i.e.,
    the runs of $u$ and $v$ on $A$ end in the exact same state $x$.
    From this, we get that for all $w$, runs of $u \cdot w$ and $v \cdot w$ on $A$ also end in the same state.
    Therefore, $u\cdot w \in \lang{A}$ if and only if $v \cdot w \in \lang{A}$.
\end{proof}

\begin{theorem}
    If $L$ is accepted by DFA $A$, then there exists a finite partitioning of the Myhill-Nerode relation on $L$.
\end{theorem}
\begin{proof}
    From lemma \ref{DFA_MN_F} we get a refining finite partitioning $f$ of the Myhill-Nerode relation on $\lang{A}$. 
    Since $L$ is accepted by $A$, $L = \lang{A}$. 
    Therefore, $f$ is a refining finite partitioning of the Myhill-Nerode relation on $L$.
    By theorem \ref{MN_MIN} there also exists a finite partition of the Myhill-Nerode relation on $L$.
\end{proof}


\subsection{Myhill-Nerode to Finite Automata}

\paragraph{}


