
\chapter{Myhill-Nerode}
\label{chap:MN}

In this chapter, we consider three additional characterizations of regular languages:

\begin{enumerate}
    \item Myhill relations,
    \item weak Nerode relations,
    \item and Nerode relations.
\end{enumerate}


We will show that these three characterizations can be used to characterize regular languages 
by proving them equivalent to the existence of a (deterministic) finite automaton.


\section{Definitions}
Before we can state the Myhill-Nerode theorem, we need a number of auxiliary definitions. We roughly follow \cite{DBLP:books/daglib/0088160}.
\begin{definition}
    Let $\equiv$ be an equivalence relation.
    The \textbf{equivalence class} of $u \in \Sigma^*$ w.r.t. $\equiv$ is the set of all $v$ such that $u \equiv v$.
    It is denoted by $[u]_\equiv$.
\end{definition}

\begin{definition}
    Let $\equiv$ be an equivalence relation.
    $\equiv$ is of \textbf{finite index} if and only if the set of $\{[u]_\equiv \; | \; u \in \Sigma^* \}$ is finite.
\end{definition}


Due to the lack of native support for quotient types in \coq, 
we formalize equivalence relations of finite index 
as surjective functions from $\Sigma^*$ to a finite type $X$.

\begin{definition}
    Let $X$ be finite.
    Let $f: \Sigma^* \mapsto X$ be surjective.
    Let $u, v \in \Sigma^*$.
    $f$ is an \textbf{equivalence relation of finite index}. 
    $u$ and $v$ are equivalent w.r.t. $f$ if and only if $f(u) = f(v)$.
    $f(u)$ is the equivalence class of $u$ w.r.t. $f$.
\end{definition}

\code{}{}{myhill_nerode_Fin_Eq_Cls}

\begin{definition}
    Let $f$ be as above. 
    Let $x \in X$. $w \in \Sigma^*$ is a \textbf{representative} of $x$ if and only if $f(w) = x$.
    Since $f$ is surjective, every $w$ has a representative. 
    We write $\crep{x}$ to denote the \textbf{canonical representative} of $x$, which we obtain by constructive choice.
\end{definition}



\code{}{}{myhill_nerode_cr}

\subsection{Myhill Relations}

\begin{definition} 
    \label{Myhill}
    Let $\equiv$ be an equivalence relation of finite index. $\equiv$ is a \textbf{Myhill}%
    \textbf{relation} \cite{DBLP:books/daglib/0088160} on $L$ if and only if

\begin{enumerate}[label=(\roman*)]

    \item\label{right_congruent}
                %\begin{definition}
        $\equiv$ is \textbf{right congruent}, i.e. for all $u, v \in \Sigma^*$ and $a \in \Sigma$,
        \begin{equation*}
            u \equiv v \Rightarrow
            u \cdot a \equiv v \cdot a.
        \end{equation*}
                %\end{definition}


    \item\label{refinement}
                %\begin{definition}
        $\equiv$ \textbf{refines} $L$, i.e. for all $u,v \in \Sigma^*$,
        \begin{equation*}
            u \equiv v \Rightarrow
            (u \in L \iff v \in L).
        \end{equation*}
                %\end{definition}

\end{enumerate}
\end{definition}

Myhill relations are commonly referred to as ``Myhill-Nerode relations''. 
In this thesis, it makes sense to split the concept of a Myhill relation from that of the Nerode relation.
Apart from the Nerode relation, which can be seen as the coarsest Myhill relation, we also define
weak Nerode relations that have no direction connection to Myhill relations.
%Additionally, Myhill relations and both kind of Nerode relations inhabit different types in our formalization.
Thus, we strictly separate the characterizations.

Mathematically, Myhill relations are required to be of finite index. 
We only formalize equivalence relations of finite index.
Thus, proving that a Myhill relation is of finite index mathematically corresponds 
to constructing a Myhill relation in our formalization.

\code{}{}{myhill_nerode_right_congruent}
\codeblock{}{}{myhill_nerode_refines}
\codeblock{}{}{myhill_nerode_Myhill_Rel}


Myhill relations correspond to the equivalence relations 
defined as the pairs of words $(u, v)$ whose runs on a DFA $A$ end in the same state. 
These relations are right congruent, refine $\lang{A}$ and are of finite index as $A$ has finitely many states. 
We will later give a formal proof of this.

\subsection{Nerode Relations}

\begin{definition}
    \label{suffix_equal}
    Let $u, v \in \Sigma^*$. Let $L$ be a language. We define the \textbf{Nerode relation} $\doteq_L$ on $L$ such that 
    %We say that $u$ and $v$ are \textbf{invariant under concatenation} w.r.t. L if and only if
    \begin{equation*}
        u \doteq_L v \iff \forall w \in \Sigma^*. \; uw \in L \Leftrightarrow vw \in L. 
    \end{equation*}
\end{definition}


The Nerode relation given above is a propositional statement in \coq. 
To proof that the Nerode relation is of finite index, 
we require an equivalence relation, 
i.e. a function $f$ from words to a finite type,
such that $f$ is equivalent to $\doteq_L$.

\code{}{}{myhill_nerode_equiv_suffix}
\codeblock{}{}{myhill_nerode_Nerode_Rel}

\begin{definition}
    \label{Weak_Nerode_Rel}
    Let $L$ be a language and let $\equiv$ be an equivalence relation. We say that $\equiv$ is a \textbf{weak Nerode relation} on $L$ if and only if
    \begin{equation*}
        \forall u, v \in \Sigma^*. \; u \equiv v \implies u \doteq_L v.
    \end{equation*}
\end{definition}


\code{}{}{myhill_nerode_suffix_equal}
\codeblock{}{}{myhill_nerode_imply_suffix}
\codeblock{}{}{myhill_nerode_Weak_Nerode_Rel}


It appears that the notion of a weak Nerode relation is not found in the literature.
We will later prove them weaker than Myhill relations, in the sense that every Myhill relation is also a weak Nerode relation.

\subsection{Myhill-Nerode Theorem}
We can now move on to the statement of the Myhill-Nerode theorem \cite{DBLP:books/daglib/0088160}.



\begin{theorem}{(Myhill-Nerode)}
    \label{Myhill-Nerode}
    Let $L$ be a language. The following four statements are equivalent:
    \begin{enumerate}
        \item there exists a deterministic finite automaton that accepts $L$;
        \item there exists a Myhill relation on $L$;
        \item there exists a weak Nerode relation on $L$;
        \item the Nerode relation on $L$ is of finite index.
    \end{enumerate}
\end{theorem}

Our proof of equivalence will have the following structure:
\begin{equation*}
    \lefteqn{
        \underbrace{
            \phantom{
                \mathit{DFA} 
                \implies
                \mathit{Myhill} 
            }
        }_\text{\circled{a}}
    }
            \mathit{DFA} 
            \implies
    \lefteqn{
        \overbrace{
            \phantom{
                \mathit{Myhill}
                \implies
                \mathit{weak}_\textrm{} \; \mathit{Nerode} 
            }
        }^\text{\circled{b}}
    }
    \mathit{Myhill}
    \implies
    \lefteqn{
        \underbrace{
            \phantom{
                \mathit{weak}_\textrm{} \; \mathit{Nerode} 
                \implies
                \mathit{Nerode}
            }
        }_\text{\circled{c}} 
    }
    \mathit{weak}_\textrm{} \; \mathit{Nerode} 
    \implies
    \lefteqn{
        \overbrace{
            \phantom{
                \mathit{Nerode}
                \implies
                \mathit{DFA}
            }
        }^\text{\circled{d}} 
    }
    \mathit{Nerode}
    \implies
    \mathit{DFA}.
\end{equation*}


We will first show \tcircled{a}, \tcircled{b}, and \tcircled{d}.
We will then give a proof of \tcircled{c}, which is the most challenging proof and formalization in this chapter.


%\section{Finite Automata, Myhill relations, and Nerode relations}
%First, we will show that, given a DFA $A$, we can construct a Myhill relation on $\lang{A}$.
%We will then show that this also enables us to construct a weak Nerode relation $\lang{A}$.
%As we have already covered the minimization of equivalence classes in the previous chapter,
%this proves that the Nerode relation is of finite index.

%Conversely, knowing that the Nerode relation on language $L$ is of finite index,  we will show that we can construct a DFA that accepts $L$.

\section{Finite Automata to Myhill relations}
We assume we are given a DFA $A$. 
We will be using the states of $A$ as equivalence classes.
Our goal is to construct a Myhill relation, for which we will need an equivalence relation of finite index.
Therefore, we first need to ensure that the mapping from words to equivalence classes is surjective.
Thus, we consider the equivalent, connected automaton $A_c=(Q_c, s_c, F_c, \delta_c)$ (Definition \ref{A_c}), which has only reachable states.
This enables us to construct a surjective function from words to the states of $A_c$.

\begin{definition}
    \label{f_M} 
    Let $u \in \Sigma^*$. 
    Let $\sigma$ be the run of $u$ on $A_c$. 
    We define $f_M: \Sigma^* \mapsto Q_c$ such that $f_M(U)$ is the last state in $\sigma$, i.e.
    \begin{equation*}
        f_M(u) := \sigma_{|\sigma|-1}.
    \end{equation*}
    Note that $f_M$ is surjective (follows directly from Lemma \ref{dfa_connected_repr}) 
    and, thus, an equivalence relation of finite index.
\end{definition}

\code{}{}{myhill_nerode_f_M}
\codeblock{}{}{myhill_nerode_f_M_surjective_head}
\codeblock{}{}{myhill_nerode_f_fin}


In order to show that $f_M$ is a Myhill relation, we prove that it fulfills Definition \ref{Myhill}.

\begin{lemma}
    \label{f_M_right_congruent}
    $f_M$ is right congruent.
\end{lemma}
\begin{proof}
    Let $u,v \in \Sigma^*$ such that ${f_M}(u) = {f_M}(v)$.
    Let $a \in \Sigma$. Since $A$ is deterministic, we get ${f_M}(ua) = {f_M}(va)$.
\end{proof}

\begin{lemma}
    \label{f_M_refines}
    $f_M$ refines $\lang{A_c}$.
\end{lemma}
\begin{proof}
    Let $u,v \in \Sigma^*$ such that ${f_M}(u) = {f_M}(v)$.
    By definition of $f_M$, the runs $u$ and $v$ on $A$ end in the same state.
    Thus, either $u$ and $v$ are both accepted, or both not accepted.
\end{proof}

\begin{theorem}
    \label{dfa_to_myhill}
    $f_M$ is a Myhill relation on $\lang{A}$.
\end{theorem}
\begin{proof}
    By Lemma \ref{dfa_connected_correct}, we have $\lang{A_c} = \lang{A}$.
    Thus, it suffices to show that $f_M$ is a Myhill relation on $\lang{A_c}$.
    This follows with Lemma \ref{f_M_right_congruent} and Lemma \ref{f_M_refines}.
\end{proof}


We only have extensional equality on $\lang{A_c}$ and $\lang{A}$ in \coq.
Thus, we first show that $f_M$ is a Myhill relation on $\lang{A_c}$.
Then, we show that we can get a Myhill relation on $\lang{A}$ from a Myhill relation on $\lang{A_c}$.

\code{}{}{myhill_nerode_dfa_to_myhill'}
\codeblock{}{}{myhill_nerode_myhill_lang_eq_head}
\codeblock{}{}{myhill_nerode_dfa_to_myhill}


This concludes the proof of step \tcircled{a}.

\section{Myhill relations to weak Nerode relations}
We show that, if there exists a Myhill relation, there also exists a weak Nerode relation.
In fact, we will prove that any Myhill relation \textit{is} a weak Nerode relation.

\begin{theorem}
    \label{myhill_suffix}
    Let $f$ be a Myhill relation on a language $L$.
    Then $f$ is a weak Nerode relation on $L$.
\end{theorem}
\begin{proof}
    Let $u, v \in \Sigma^*$ such that $u =_f v$. We have to show that for all $w \in \Sigma^*$, $u w \in L \Leftrightarrow v w \in L$.
    By induction on $w$.
    \begin{enumerate}
        \item For $w = \varepsilon$, we get $u \in L \Leftrightarrow v \in L$ as $f$ refines $L$.
        \item For $w = aw'$, we get $ua =_f va$ by congruence of $f$ and thus, by inductive hypothesis, $uaw' \in L \Leftrightarrow vaw' \in L$.
    \end{enumerate}
\end{proof}

\code{}{}{myhill_nerode_myhill_suffix_head}
\codeblock{}{}{myhill_nerode_myhill_to_weak_nerode}


This concludes step \tcircled{b} of Theorem \ref{Myhill-Nerode}.

\section{Nerode relations to Finite Automata}
We prove step \tcircled{d} of Theorem \ref{Myhill-Nerode}. 
If the Nerode relation on a language $L$ is of finite index, we can construct a DFA that accepts $L$.
The construction is very straight-forward and uses the equivalence classes of the Nerode relation as the set of states for the automaton.

\begin{definition}
    \label{nerode_to_dfa} 
    Let $L$ be a language. Let $X$ be a finite type. Let $f: \Sigma^* \mapsto X$ be the equivalence relation which proves that the Nerode relation on $L$ is of finite index.
    We construct DFA $A$ such that
\begin{eqnarray*}
    s & := & f(\varepsilon) \\
    F & := & \{ x | x \in X \wedge \crep{x} \in L \} \\
    \delta & := & \{ (x, a, f(\crep{x} a)) \; | \; x \in X, a \in \Sigma \} \\
    A &:=&  (X, s, F, \delta).
\end{eqnarray*}
\end{definition}

\code{}{}{myhill_nerode_nerode_to_dfa}


In order to show that $A$ accepts the language $L$, we first need to connect runs on $A$ to the equivalence classes, i.e. the range of $f$.
The following lemma gives a direct connection.

\begin{lemma}
    \label{nerode_to_dfa_run}
    Let $w \in \Sigma^*$. Let $\sigma$ be the run of $w$ on $A$ starting in $s$. We have that the last state of $\sigma$ is the equivalence class of $w$, i.e. 
    \begin{equation*}
        \sigma_{|\sigma|-1} = f(w).
    \end{equation*}
\end{lemma}
\begin{proof}
    We proceed by induction on $w$ from right to left.
    \begin{enumerate}
        \item For $w = \varepsilon$ we have $s = f(\varepsilon)$.
        \item For $w = w'a$ we know that the run of $w'$ on $A$ starting in $s$ ends in $f(w')$. 
            It remains to show that $(f(w'), a, f(w)) \in \delta$.
            We have $\crep{f(w')}a =_f w$, i.e. $f(\crep{f(w')}a) = f(w)$ by definition of $f$.
            Thus, it suffices to show $(f(w'), a, f(\crep{f(w')}a)) \in \delta$, which holds by definition of $\delta$.
    \end{enumerate}
\end{proof}

\begin{theorem}
    \label{nerode_to_dfa_correct}
    $A$ accepts $L$, i.e. $\lang{A} = L$.
\end{theorem}
\begin{proof}
    Let $w \in \Sigma^*$. Let $\sigma$ be the run of $w$ on $A$ starting in $s$. 
    $w$ is accepted if and only if $\sigma_{|\sigma|-1} \in F$, i.e. if and only if $\crep{\sigma_{|\sigma|-1}} \in L$.
    We have $w =_f \crep{\sigma_{|\sigma|-1}}$ and therefore $w \in L \Leftrightarrow \crep{\sigma_{|\sigma|-1}} \in L$.
    Thus $w$ is accepted if and only if $w \in L$.
\end{proof}

The resulting automaton is minimal, i.e. there exists no other automaton that accepts the same language and has less states than $A$.

This concludes step \tcircled{d} of Theorem \ref{Myhill-Nerode}.

\section{Minimizing Equivalence Classes}
Finally, we prove that if there is a weak Nerode relation on a language $L$, the Nerode relation is of finite index.
This proves step \tcircled{c} of the Theorem \ref{Myhill-Nerode} and thus complets its proof.
For this purpose, we employ a table-filling algorithm \cite{Huffman1954161} to find indistinguishable states under the Myhill-Nerode relation. 
%However, we do not rely on an automaton, as is usually done. 
This algorithm was originally intended to be used on automata. 
It identifies (un)distinguishable states based on their successors.
We use the finite type $X$, i.e., the equivalence classes, instead of states.


For the remainder of this section, we assume we are given a language $L$ and a weak Nerode relation $f_0$. 


We employ a fixed-point construction to find equivalence classes that are $\doteq_L$-equivalent.
In each step, we add those equivalences classes that are distinguishable based on equivalence classes that were distinguishable in the previous step.
The initial set of distinguishable equivalence classes are distinguishable by the inclusion of their canonical representative in $L$. 
We denote this initial set $\mathit{dist_0}$.
\begin{equation*}
    dist_0 := \{ (x,y)  \in F \times F \, | \, \crep{x} \in L \Leftrightarrow \crep{y} \notin L\}.
\end{equation*}

\code{}{}{myhill_nerode_distinguishable}
\codeblock{}{}{myhill_nerode_dist0}


To find more distinguishable equivalence classes, we have to identify equivalence classes that ``lead'' to distinguishable equivalence classes. 
In analogy to the minimization procedure on automata, we define successors of equivalence classes with respect to a single character.
The intuition is that two states are distinguishable if they are succeeded by a pair of distinguishable states.
Conversely, if a pair of states is not distinguishable, then their predecessors can not be distinguished by those states.
Thus, two states are undistinguishable if none of their succeeding pairs of states are distinguishable.


\begin{definition}
    %We say that an equivalence class $x$ \textbf{transitions} to $y$ with $a \in \Sigma$ if and only if
    Let $x,y \in X$ and $a \in \Sigma$.
    We define $\mathit{succ_a}$ and $\mathit{psucc_a}$. $\mathit{succ_a}(x) := {f_0}(\crep{x}\cdot a)$ and $\mathit{psucc_a}(x,y) := (\mathit{succ_a}(x), \mathit{succ_a}(y))$.
\end{definition}
\code{}{}{myhill_nerode_succ}
\codeblock{}{}{myhill_nerode_psucc}

The fixed-point algorithm tries to extend the set of distinguishable equivalence classes by looking for a pair of equivalence classes that transitions to a pair of distinguishable equivalence classes. 
Given a set of pairs of equivalence classes $\mathit{dist}$, 
we define the set of pairs of distinguishable equivalence classes that have successors in $\mathit{dist}$ as
\begin{equation*}
    \mathit{dist_S}(\mathit{dist}) := \{ (x,y) \; | \; \exists a. \, \mathit{psucc_a}(x,y) \in \mathit{dist}\}.
\end{equation*}
\code{}{}{myhill_nerode_distS}

\begin{definition}
    \label{one_step_dist}
    Let $\mathit{dist}$ be a subset of $X \times X$. We define $\mathit{one\mhyphen step\mhyphen dist}$ such that
    \begin{equation*}
        \mathit{one\mhyphen step\mhyphen dist}(\mathit{dist}) := \mathit{dist_0} \cup \mathit{dist} \cup \mathit{distinct_S}(\mathit{dist}).
    \end{equation*}
\end{definition}
\code{}{}{myhill_nerode_one_step_dist}


\begin{lemma}
    \label{dist_monotone}
    $\mathit{one\mhyphen step\mhyphen dist}$ is monotone and has a fixed-point.
\end{lemma}
\begin{proof}
    Monotonicity follows directly from the monotonicity of $\cup$. 
    The number of sets in $X \times X$ is finite. 
    Therefore, $\mathit{one\mhyphen step\mhyphen dist}$ has a fixed point.
    We iterate $\mathit{one\mhyphen step\mhyphen dist}$ $|X*X|+1 = |X|^2+1$ times on the empty set.
    Since there can only ever be $|X*X|$ items in the result of $\mathit{one\mhyphen step\mhyphen dist}$, 
    we will find the fixed point in no more than $|X*X|+1$ steps.
    
    Let \textit{\textbf{distinct}} be the fixed point of $\mathit{one\mhyphen step\mhyphen dist}$ and let it be denoted by $\not\cong$. 
We write \textit{\textbf{equiv}} for the complement of \textit{distinct} and denote it $\cong$.
\end{proof}
\code{}{}{base_lfp}
\codeblock{}{}{myhill_nerode_distinct}



We now show that $\cong$ is equivalent to the Nerode relation. 
Formally, this means constructing a function that fulfills our definition
of an equivalence relation of finite index.
Then, we prove that this equivalence relation is equivalent to the the Nerode relation.
First, we will prove that $\cong$ is an equivalence relation.
Then, we will prove it equivalent to the Nerode relation in two separate steps, 
since the two directions require different strategies.

\begin{lemma}
    \label{equiv_equiv}
    $\cong$ is an equivalence relation.
\end{lemma}
\begin{proof}
    We first state transitivity of $\cong$ in terms of $\not\cong$:% and call this property of $\not\cong$ \textbf{complementary transitivity}:
    \begin{equation}
        \tag{*}
        \label{trans}
        \forall x,y,z \in X. \; \neg (x \not\cong y) \implies \neg (y \not\cong z) \implies \neg (x \not\cong z).
    \end{equation}
    It suffices to show that $\mathit{distinct}$ is anti-reflexive, symmetric and fulfills (\ref{trans}).
    Note that complementary transitivity, anti-reflexivity and symmetry are closed under union.
    We proceed by fixed-point induction.
    \begin{enumerate}
        \item For $\mathit{one\mhyphen step\mhyphen dist}(\mathit{dist}) = \emptyset$ we have anti-reflexivity, symmetry and (\ref{trans}) %complementary transitivity
            by the properties of $\emptyset$.
        \item For $\mathit{one\mhyphen step\mhyphen dist}(\mathit{dist}) = \mathit{dist'}$ 
            we have $\mathit{dist}$ anti-reflexive, symmetric and (\ref{trans}). %complementarily transitive.
            It remains to show that
            $\mathit{dist_0}$ 
            and $\mathit{distinct_S}(\mathit{dist})$ 
            are anti-reflexive, symmetric and fulfill (\ref{trans}). 
            
            $\mathit{dist_0}$ is anti-reflexive, symmetric and fulfills (\ref{trans})  by definition. 

            $\mathit{distinct_S}(\mathit{dist})$ can be seen as an intersection 
            of a symmetric subset of $X \times X$ defined by $\mathit{psucc_a}$ and  $\mathit{dist}$, 
            the latter of which is anti-reflexive, symmetric and fulfills (\ref{trans}).
            Thus, $\mathit{distinct_S}(\mathit{dist})$ is anti-reflexive, symmetric and fulfills (\ref{trans}). % complementarily transitive.
            
            Therefore, $\mathit{dist'}$ is anti-reflexive, symmetric and fulfills (\ref{trans}). %complementarily transitive.
    \end{enumerate}
\end{proof}

\code{}{}{myhill_nerode_equiv_refl_head}
\codeblock{}{}{myhill_nerode_equiv_sym_head}
\codeblock{}{}{myhill_nerode_equiv_trans_head}

\begin{lemma}
    \label{equiv_final}
    Let $u,v \in \Sigma^*$. 
    ${f_0}(u) \cong {f_0}(v) \implies u \doteq_L v$.
\end{lemma}
\begin{proof}
    Let $w \in \Sigma^*$. We then show the contraposition of the claim: 
    \begin{equation*}
        uw \in L \not\Leftrightarrow vw \in L \implies {f_0}(u) \not\cong {f_0}(v).
    \end{equation*}    
    By induction on $w$.
    \begin{enumerate}
        \item For $w = \varepsilon$ we have $u \in L \not\Leftrightarrow v \in L$ which gives us $({f_0}(u),{f_0}(v)) \in dist_0$.
            Thus, ${f_0}(u) \not\cong {f_0}(v)$.
        \item For $w = aw'$ we have $uaw \in L \not\Leftrightarrow vaw \in L$.
            We have to show ${f_0}(u) \not\cong {f_0}(v)$, i.e. $({f_0}(u), {f_0}(v)) \in \mathit{distinct}$.
            By definition of $\mathit{distinct}$, it suffices to show $({f_0}(u), {f_0}(v)) \in \mathit{one\mhyphen step\mhyphen dist}(\mathit{distinct})$.

             For this, we prove $({f_0}(u), {f_0}(v)) \in \mathit{distinct_S}(\mathit{distinct})$. 
             By $uaw \in L \not\Leftrightarrow vaw \in L$ we have $({f_0}(\crep{u}a), {f_0}(\crep{v}a)) \in dist_0$.

             It remains to show that ${f_0}(\crep{u}a) \not\cong {f_0}(\crep{v}a)$ which we get by inductive hypothesis.
             For this, we need to show that $\crep{u}aw \in L \not\Leftrightarrow \crep{v}aw \in L$.

             By the properties of $f$, we get $\crep{u}aw \in L \Leftrightarrow uaw \in L$ and $\crep{v}aw \in L \Leftrightarrow vaw \in L$.
             Thus, $\crep{u}aw \in L \not\Leftrightarrow \crep{v}aw$.
             
    \end{enumerate}
\end{proof}


\begin{lemma}
    \label{distinct_final}
    Let $u,v \in \Sigma^*$. 
    If ${f_0}(u) \not\cong {f_0}(v)$, then $u$ and $v$ are \textbf{not} equivalent wr.t. the Merode relation, i.e. ${f_0}(u) \not\cong {f_0}(v) \implies u \not\doteq_L v$.
\end{lemma}
\begin{proof}
    We do a fixed-point induction.
    \begin{enumerate}
        \item For $\mathit{one\mhyphen step\mhyphen dist}(\mathit{dist}) = \emptyset$ we have $({f_0}(u), {f_0}(v)) \in \emptyset$ and thus a contradiction. 
        \item For $\mathit{one\mhyphen step\mhyphen dist}(\mathit{dist}) = \mathit{dist'}$ we have $({f_0}(u), {f_0}(v)) \in \mathit{dist'}$. 
            We do a case distinction on $\mathit{dist'}$.
            \begin{enumerate}
                \item $({f_0}(u), {f_0}(v)) \in \mathit{dist_0}$.
                    We have $u \in L \not\Leftrightarrow v \in L$. 
                    Thus, $u \not\doteq_L v$ as witnessed by $w=\varepsilon$.
                \item $({f_0}(u), {f_0}(v)) \in \mathit{dist}$. 
                    By inductive hypothesis, $u \not\doteq_L v$.
                \item $({f_0}(u), {f_0}(v)) \in \mathit{distinct_S}(\mathit{dist})$.
                    We have $a \in \Sigma$ with $\mathit{psucc_a}({f_0}(u), {f_0}(v))) \in \mathit{dist}$.
                    By inductive hypothesis, we get $\crep{u}a \not\doteq_L \crep{v}a$ 
                    as witnessed by $w \in \Sigma^*$ 
                    such that $\crep{u}aw \in L \not\Leftrightarrow \crep{v}aw \in L$.

                    By the properties of $f$, we get $\crep{u}aw \in L \Leftrightarrow uaw \in L$ and $\crep{v}aw \in L \Leftrightarrow vaw \in L$.
                    Thus, we have $u \not\doteq_L v$ as witnessed by $aw$.
            \end{enumerate}
    \end{enumerate}
\end{proof}


\begin{corollary}
    \label{equivP}
    Let $u, v \in \Sigma^*$. We have that
    \begin{equation*}
        {f_0}(u) \cong {f_0}(v) \iff u \doteq_L v.
    \end{equation*}
\end{corollary}
\begin{proof}
    Follows immediately with Lemma \ref{equiv_final} and Lemma \ref{distinct_final}. 
\end{proof}

\code{}{}{myhill_nerode_equiv_suffix_equal_head}
\codeblock{}{}{myhill_nerode_distinct_not_suffix_equal_head}
\codeblock{}{}{myhill_nerode_equivP_head}

\begin{definition}
    \label{f_min}
    Let $w \in \Sigma^*$. We define
    %Then $\mathit{f_{min}}$ is defined such that 
    \begin{equation*}
        \mathit{f_{min}}(w) := \{ x \; | \; x \in X, \; {f_0}(w) \cong x \}.
    \end{equation*}
    Note that the range of $\mathit{f_{min}}$ is finite (since $X$ is finite) and does not contain the empty set (due to reflexivity of $\cong$).
\end{definition}

\begin{lemma}
    \label{f_min_surjective}
    $\mathit{f_{min}}$ is surjective.
\end{lemma}
\begin{proof}
    Let $s \in range(\mathit{f_{min}})$. 
    Since $s \neq \emptyset$, 
    there exists $x \in X$ such that $x \in s$.
    We have ${f_0}(x) = {f_0}(\crep{x}$) and therefore ${f_0}(x) \cong {f_0}(\crep{x})$ by reflexivity of $\cong$.
    Thus, ${f_0}(\crep{x}) = s$ since  $\mathit{f_{min}}(x) = \mathit{f_{min}}(\crep{x}) = s$.
\end{proof}

\begin{lemma}
    \label{f_minP}
    For all $u,v \in \Sigma^*$ we we have 
    \begin{equation*}
        \mathit{f_{min}}(u) = \mathit{f_{min}}(v) \iff {f_0}(u) \cong {f_0}(v).
    \end{equation*}
\end{lemma}

\begin{proof}
    ``$\Rightarrow$''
    We have $\mathit{f_{min}}(u) = \mathit{f_{min}}(v)$ and thus ${f_0}(u) \cong {f_0}(v)$.

    ``$\Leftarrow$''
    We have ${f_0}(u) \cong {f_0}(v)$. 
    Let $x \in X$.
    It suffices to show that ${f_0}(u) \cong x$ if and only if ${f_0}(v) \cong x$.
    This follows with symmetry and transitivity of $\cong$.
\end{proof}

\begin{lemma}
    \label{f_min_correct}
    $\mathit{f_{min}}$ is equivalent to the Nerode relation, i.e. $\mathit{f_{min}}$ is surjective and for all $u,v \in \Sigma^*$ we have
    \begin{equation*}
        \mathit{f_{min}}(u) = \mathit{f_{min}}(v) \iff u \doteq_L v.
    \end{equation*}
\end{lemma}

\begin{proof}
    We have proven surjectivity in Lemma \ref{f_min_surjective}. 
    By Lemma \ref{f_minP} we have $\mathit{f_{min}}(u) = \mathit{f_{min}}(v)$ if and only if ${f_0}(u) \cong {f_0}(v)$.
    By corollary \ref{equivP} we have ${f_0}(u) \cong {f_0}(v)$ if and only if $u \doteq_L v$.
    Thus, $\mathit{f_{min}}(u) = \mathit{f_{min}}(v)$ if and only if $u \doteq_L v$.
\end{proof}


The formalization of $\mathit{f_{min}}$ is slightly more involved than the mathematical construction. 
We first need to define the finite type of $\mathit{f_{min}}$'s range, 
which we do by enumerating all possible values of $\mathit{f_{min}}$.

\code{}{}{myhill_nerode_equiv_repr}
\codeblock{}{}{myhill_nerode_X_min}
\codeblock{}{}{myhill_nerode_f_min}


We then prove lemmas \ref{f_min_surjective}, \ref{f_minP} and Theorem \ref{f_min_correct} which are consequential and straight-forward.

\code{}{}{myhill_nerode_f_min_surjective_head}
\codeblock{}{}{myhill_nerode_f_minP_head}
\codeblock{}{}{myhill_nerode_f_min_correct_head}
\codeblock{}{}{myhill_nerode_f_min_fin}


We can now state the result of this section. 

\begin{corollary}
    The Nerode relation is of finite index.
\end{corollary}
\begin{proof}
    This follows directly from Lemma \ref{equiv_equiv} and Lemma \ref{f_min_correct}.
\end{proof}

\code{}{}{myhill_nerode_weak_nerode_to_nerode}


This concludes step \tcircled{c} of Theorem \ref{Myhill-Nerode} and, thus, this chapter. 
The characterizations presented in this chapter are very compact, mathematically.
They also lend themselves very well to formalization.


