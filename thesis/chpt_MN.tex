
\chapter{Myhill-Nerode}
\label{chap:MN}

In this chapter, we consider three additional characterizations of regular languages:

\begin{enumerate}
    \item Myhill relations,
    \item weak Nerode relations,
    \item and Nerode relations.
\end{enumerate}

\paragraph{}
We will show that these three concepts can be used to characterize regular languages 
by proving them equivalent to the existence of a (deterministic) finite automaton.
Our proof of equivalence will have the following structure:
\begin{equation*}
    \lefteqn{
        \underbrace{
            \phantom{
                \mathit{DFA} 
                \implies
                \mathit{Myhill} 
            }
        }_\text{\circled{a}}
    }
            \mathit{DFA} 
            \implies
    \lefteqn{
        \overbrace{
            \phantom{
                \mathit{Myhill}
                \implies
                \mathit{weak}_\textrm{} \; \mathit{Nerode} 
            }
        }^\text{\circled{b}}
    }
    \mathit{Myhill}
    \implies
    \lefteqn{
        \underbrace{
            \phantom{
                \mathit{weak}_\textrm{} \; \mathit{Nerode} 
                \implies
                \mathit{Nerode}
            }
        }_\text{\circled{c}} 
    }
    \mathit{weak}_\textrm{} \; \mathit{Nerode} 
    \implies
    \lefteqn{
        \overbrace{
            \phantom{
                \mathit{Nerode}
                \implies
                \mathit{DFA}
            }
        }^\text{\circled{d}} 
    }
    \mathit{Nerode}
    \implies
    \mathit{DFA}.
\end{equation*}

\paragraph{}
We will first give a proof of \tcircled{c}, which is the most challenging proof and formalization in this chapter.
We will then show \tcircled{a}, \tcircled{b}, and \tcircled{d}.

%
%We will prove the equivalence of the following four%
%\footnote{The Myhill-Nerode theorem usually consists only of the equivalence of 
%\ref{
%}
%statements. Let $L$ be a language.
%\begin{enumerate}
%    \item \label{mn_thm_fa} There exists a finite automaton $A$ such that $\lang{A} = L$.
%    \item \label{mn_thm_myhill} There exists a Myhill relation on $L$.
%    \item \label{mn_thm_weak_nerode} There exists a weak Nerode relation on $L$.
%    \item \label{mn_
%\end{enumerate}


\section{Definition}

The following definitions (roughly following \cite{DBLP:books/daglib/0088160}) will lead us to the statement of the Myhill-Nerode theorem.
\todo{put Theorem s/w}
%Let $\equiv$ be an equivalence relation on $\Sigma^*$. Let $L$ be a language over $\Sigma$.

\begin{definition}
    The \textbf{equivalence class} of $u \in \Sigma^*$ w.r.t. $\equiv$ is the set of all $v$ such that $u \equiv v$.
    It is denoted by $[u]_\equiv$.
\end{definition}

\begin{definition}
    $\equiv$ is of \textbf{finite index} if and only if the set of $\{[u]_\equiv \; | \; u \in \Sigma^* \}$ is finite.
\end{definition}

\paragraph{}
Due to the lack of native support for quotient types in \coq, 
we formalize equivalence relations of finite index 
as surjective functions from $\Sigma^*$ to a finite type $X$.

\begin{definition}
    \label{equiv_f}
    Let $f: \Sigma^* \mapsto X$ be surjective.
    The relation $\equiv_f$ is defined such that for all $u,v \in \Sigma^*$ 
    \begin{equation*}
        u \equiv_f v \iff f(u) = f(v).
    \end{equation*}
    For all $w \in \Sigma^*$, $f(w)$ can be seen as an equivalence class of $\equiv_f$.
\end{definition}

\paragraph{}
It is easy to see that $\equiv_f$ is an equivalence relation. 
Furthermore, from the finiteness of $X$, it follows that $\equiv_f$ is of finite index.
%For our purposes, only surjective functions are relevant.

\code{}{}{myhill_nerode_Fin_Eq_Cls}

\begin{definition}
    Let $f$ be as above. 
    Let $x \in X$. $w \in \Sigma^*$ is a \textbf{representative} of $x$ if and only if $f(w) = x$.
    We write $\crep{x}$ to denote the \textbf{canonical representative} of $x$, which we obtain by constructive choice.
\end{definition}

\paragraph{}

\code{}{}{myhill_nerode_cr}

\subsection{Myhill Relations}

\begin{definition} Let $\equiv$ be an equivalence relation. $\equiv$ is a \textbf{Myhill}%
    \footnote{Myhill relations are commonly referred to as ``Myhill-Nerode relations''. 
    In this thesis, it makes sense to split the concept of a Myhill relation from that of the Nerode relation.}
    \textbf{relation} \cite{DBLP:books/daglib/0088160} if and only if

\begin{enumerate}[label=(\roman*)]

    \item\label{right_congruent}
                %\begin{definition}
        $\equiv$ is \textbf{right congruent}, i.e. for all $u, v \in \Sigma^*$ and $a \in \Sigma$,
        \begin{equation*}
            u \equiv v \Rightarrow
            u \cdot a \equiv v \cdot a.
        \end{equation*}
                %\end{definition}


    \item\label{refinement}
                %\begin{definition}
        $\equiv$ \textbf{refines} $L$, i.e. for all $u,v \in \Sigma^*$,
        \begin{equation*}
            u \equiv v \Rightarrow
            (u \in L \iff v \in L).
        \end{equation*}
                %\end{definition}

    \item\label{finite_index}
                %\begin{definition}
        $\equiv$ is of \textbf{finite index}.
\end{enumerate}
\end{definition}

\paragraph{}
Building on our formalization of equivalence relations of finite index, 
we only need to give formalizations of \ref{right_congruent} and \ref{refinement}.

\code{}{}{myhill_nerode_right_congruent}
\codeblock{}{}{myhill_nerode_refines}
\codeblock{}{}{myhill_nerode_Myhill_Rel}

\paragraph{}
Myhill relations correspond to the equivalence relations 
defined as the pairs of words $(u, v)$ whose runs on a DFA $A$ end in the same state. 
These relations are right congruent, refine $\lang{A}$ and are of finite index as $A$ has finitely many states. 
We will later give a formal proof of this.

\subsection{Nerode Relations}

\begin{definition}
    \label{equal_suffix}
    Let $u, v \in \Sigma^*$. We define the \textbf{Nerode relation} $\doteq_L$ on a language $L$ such that 
    %We say that $u$ and $v$ are \textbf{invariant under concatenation} w.r.t. L if and only if
    \begin{equation*}
        u \doteq_L v \iff \forall w \in \Sigma^*. \; uw \in L \Leftrightarrow vw \in L. 
    \end{equation*}
\end{definition}

\code{}{}{myhill_nerode_equiv_suffix}
\codeblock{}{}{myhill_nerode_Nerode_Rel}

\begin{definition}
    \label{Weak_Nerode_Rel}
    Let $\equiv$ be an equivalence relation. We say that $\equiv$ is a \textbf{weak Nerode relation} if and only if
    \begin{equation*}
        \forall u, v \in \Sigma^*. \; u \equiv v \implies u \doteq_L v.
    \end{equation*}
\end{definition}


\code{}{}{myhill_nerode_equal_suffix}
\codeblock{}{}{myhill_nerode_imply_suffix}
\codeblock{}{}{myhill_nerode_Weak_Nerode_Rel}

\paragraph{}
The notion of a weak Nerode relation is not found in the literature.
We will later prove them weaker than Myhill relations, in the sense that every Myhill relation is also a weak Nerode relation.



\section{Minimizing Equivalence Classes}

We will prove that if there is a weak Nerode relation on a language $L$, the Nerode relation is of finite index.  %can be converted into Nerode relations.
For this purpose, we employ a table-filling algorithm \cite{DBLP:books/daglib/0011126} to find indistinguishable states under the Myhill-Nerode relation. 
However, we do not rely on an automaton, as is usually done. 
In fact, we use the finite type $X$, i.e., the equivalence classes, instead of states.
For the reminder of this section, we assume we are given a language $L$ and a weak Nerode relation $f_0$. 

\paragraph{}
We employ a fixed-point construction to find equivalence classes that are equivalent in the sense of $\doteq_L$.
We are starting with a equivalence relation of finite index, that $\doteq_L$ 
In each step, we add those equivalences classes that are distinguishable based on previously distinguishable equivalence classes.
The initial set of distinguishable equivalence classes are distinguishable by the inclusion of their canonical representative in $L$. 
We denote this initial set $\mathit{dist_0}$.
\begin{equation*}
    dist_0 := \{ (x,y)  \in F \times F \, | \, \crep{x} \in L \Leftrightarrow \crep{y} \notin L\}.
\end{equation*}

\code{}{}{myhill_nerode_distinguishable}
\codeblock{}{}{myhill_nerode_distinct0}

\paragraph{}
To find more distinguishable equivalence classes, we have to identify equivalence classes that ``lead'' to distinguishable equivalence classes. 
In allusion to the minimization procedure on automata, we define successors of equivalence classes.
The intuition is that two states are distinguishable if they are succeeded by distinguishable states.
Conversely, if a pair of states is not distinguishable, then their predecessors can not be distinguished by those states.
Thus, two states are undistinguishable, i.e. equivalent, if none of their succeeding pairs of states are distinguishable.


\begin{definition}
    %We say that an equivalence class $x$ \textbf{transitions} to $y$ with $a \in \Sigma$ if and only if
    Let $x,y \in X$ and $a \in \Sigma$.
    We define $\mathit{succ_a}$ and $\mathit{psucc_a}$. $\mathit{succ_a}(x) := \crep{x}\cdot a$ and $\mathit{psucc_a}(x,y) := (\mathit{succ_a}(x), \mathit{succ_a}(y))$.
\end{definition}

The fixed-point algorithm tries to extend the set of distinguishable equivalence classes by looking for a pair of equivalence classes that transitions to a pair of distinguishable equivalence classes. 
Given a set of pairs of equivalence classes $\mathit{dist}$, 
we define the set of pairs of distinguishable equivalence classes that are succeeded by pairs in $\mathit{dist}$ as
\begin{equation*}
    \mathit{distinct_S}(\mathit{dist}) := \{ (x,y) \; | \; \exists a. \, \mathit{psucc_a}(x,y) \in \mathit{dist}\}.
\end{equation*}

\begin{definition}
    \label{unnamed}
    Let $\mathit{dist}$ be a subset of $X \times X$. We define $\mathit{unnamed}$ such that
    \begin{equation*}
        \mathit{unnamed}(\mathit{dist}) := \mathit{dist_0} \cup \mathit{dist} \cup \mathit{distinct_S}(\mathit{dist}).
    \end{equation*}
\end{definition}


\code{}{}{myhill_nerode_succ}

\codeblock{}{}{myhill_nerode_psucc}

\codeblock{}{}{myhill_nerode_distinctS}

\codeblock{}{}{myhill_nerode_unnamed}

\begin{lemma}
    \label{dist_monotone}
    $unnamed$ is monotone and has a fixed-point.
\end{lemma}
\begin{proof}
    Monotonicity follows directly from the monotonicity of $\cup$. 
    The number of sets in $X \times X$ is finite. 
    Therefore, $unnamed$ has a fixed point.
    We iterate $\mathit{unnamed}$ $|X*X|+1 = |X|^2+1$ times on the empty set.
    Since there can only ever be $|X*X|$ items in the result of $\mathit{unnamed}$, 
    we will find the fixed point in no more than $|X*X|+1$ steps.
    \paragraph{}
Let \textit{\textbf{distinct}} be the fixed point of $unnamed$ and let it be denoted by $\not\cong$. 
We write \textit{\textbf{equiv}} for the complement of \textit{distinct} and denote it $\cong$.
\end{proof}

\paragraph{}
We now show that $\cong$ is equivalent to the Nerode relation. 
First, we will prove that $\cong$ is an equivalence relation.
Then, we will prove it equivalent to the Nerode relation in two separate steps, 
since the two directions require different strategies.

\begin{lemma}
    \label{equiv_refl}
    $\cong$ is an equivalence relation.
\end{lemma}
\begin{proof}
    We first state transitivity of $\cong$ in terms of $\not\cong$ and call this property of $\not\cong$ \textbf{complementary transitivity}:
    \begin{equation*}
        \forall x,y,z \in X. \; \neg (x \not\cong y) \implies \neg (y \not\cong z) \implies \neg (x \not\cong z).
    \end{equation*}
    It suffices to show that $\mathit{distinct}$ is anti-reflexive, symmetric and complementarily transitive.
    Note that complementary transitivity, anti-reflexivity and symmetry are closed under union.
    We do a fixed-point induction.
    \begin{enumerate}
        \item For $\mathit{unnamed}(\mathit{dist}) = \emptyset$ we have anti-reflexivity, symmetry and complementary transitivity
            by the properties of $\emptyset$.
        \item For $\mathit{unnamed}(\mathit{dist}) = \mathit{dist'}$ 
            we have $\mathit{dist}$ anti-reflexive, symmetric and complementarily transitive.
            It remains to show that
            $\mathit{dist_0}$ 
            and $\mathit{distinct_S}(\mathit{dist})$ 
            are anti-reflexive, symmetric and complementarily transitive.
            
            $\mathit{dist_0}$ is anti-reflexive, symmetric and complementarily transitive by definition. 

            $\mathit{distinct_S}(\mathit{dist})$ can be seen as an intersection 
            of a symmetric subset of $X \times X$ defined by $\mathit{psucc_a}$ and the anti-reflexive, symmetric and complementarily transitive $\mathit{dist}$.
            Thus, $\mathit{distinct_S}(\mathit{dist})$ is anti-reflexive, symmetric and complementarily transitive.
            
            Therefore, $\mathit{dist'}$ is anti-reflexive, symmetric and complementarily transitive.
    \end{enumerate}
\end{proof}

\code{}{}{myhill_nerode_equiv_refl_head}
\codeblock{}{}{myhill_nerode_equiv_sym_head}
\codeblock{}{}{myhill_nerode_equiv_trans_head}

\begin{lemma}
    \label{equiv_final}
    Let $u,v \in \Sigma^*$. 
    $u \cong_{f_0} v \implies u \doteq_L v$.
\end{lemma}
\begin{proof}
    Let $w \in \Sigma^*$. We then show the contraposition of the claim: 
    \begin{equation*}
        uw \in L \not\Leftrightarrow vw \in L \implies u \not\cong_{f_0} v.
    \end{equation*}    
    Induction on $w$ and generalize over $u$ and $v$.
    \begin{enumerate}
        \item For $w = \varepsilon$ we have $u \in L \not\Leftrightarrow v \in L$ which gives us $({f_0}(u),{f_0}(v)) \in dist_0$.
            Thus, $u \not\cong_{f_0} v$.
        \item For $w = aw'$ we have $uaw \in L \not\Leftrightarrow vaw \in L$.
            We have to show $u \not\cong_{f_0} v$, i.e. $({f_0}(u), {f_0}(v)) \in \mathit{distinct}$.
            By definition of $\mathit{distinct}$, it suffices to show $({f_0}(u), {f_0}(v)) \in \mathit{unnamed}(\mathit{distinct})$.

             For this, we prove $({f_0}(u), {f_0}(v)) \in \mathit{distinct_S}(\mathit{distinct})$. 
             By $uaw \in L \not\Leftrightarrow vaw \in L$ we have $({f_0}(\crep{u}a), {f_0}(\crep{v}a)) \in dist_0$.

             It remains to show that $\crep{u}a \not\cong_{f_0} \crep{v}a$ which we get by inductive hypothesis.
             For this, we need to show that $\crep{u}aw \in L \not\Leftrightarrow \crep{v}aw \in L$.

             By the properties of $f$, we get $\crep{u}aw \in L \Leftrightarrow uaw \in L$ and $\crep{v}aw \in L \Leftrightarrow vaw \in L$.
             Thus, $\crep{u}aw \in L \not\Leftrightarrow \crep{v}aw$.
             
    \end{enumerate}
\end{proof}


\begin{lemma}
    \label{distinct_final}
    Let $u,v \in \Sigma^*$. 
    If $u \not\cong_{f_0} v$, then $u$ and $v$ are \textbf{not} invariant under concatenation, i.e. $u \not\cong_{f_0} v \implies u \not\doteq_L v$.
\end{lemma}
\begin{proof}
    We do a fixed-point induction.
    \begin{enumerate}
        \item For $\mathit{unnamed}(\mathit{dist}) = \emptyset$ we have $({f_0}(u), {f_0}(v)) \in \emptyset$ and thus a contradiction. 
        \item For $\mathit{unnamed}(\mathit{dist}) = \mathit{dist'}$ we have $({f_0}(u), {f_0}(v)) \in \mathit{dist'}$. 
            We do a case distinction on $\mathit{dist'}$.
            \begin{enumerate}
                \item $({f_0}(u), {f_0}(v)) \in \mathit{dist_0}$.
                    We have $u \in L \not\Leftrightarrow v \in L$. 
                    Thus, $u \not\doteq_L v$ as witnessed by $w=\varepsilon$.
                \item $({f_0}(u), {f_0}(v)) \in \mathit{dist}$. 
                    By inductive hypothesis, $u \not\doteq_L v$.
                \item $({f_0}(u), {f_0}(v)) \in \mathit{distinct_S}(\mathit{dist})$.
                    We have $a \in \Sigma$ with $\mathit{psucc_a}({f_0}(u), {f_0}(v))) \in \mathit{dist}$.
                    By inductive hypothesis, we get $\crep{u}a \not\doteq_L \crep{v}a$ 
                    as witnessed by $w \in \Sigma^*$ 
                    such that $\crep{u}aw \in L \not\Leftrightarrow \crep{v}aw \in L$.

                    By the properties of $f$, we get $\crep{u}aw \in L \Leftrightarrow uaw \in L$ and $\crep{v}aw \in L \Leftrightarrow vaw \in L$.
                    Thus, we have $u \not\doteq_L v$ as witnessed by $aw$.
            \end{enumerate}
    \end{enumerate}
\end{proof}


\begin{corollary}
    \label{equivP}
    Let $u, v \in \Sigma^*$. We have that
    \begin{equation*}
        u \cong_{f_0} v \iff u \doteq_L v.
    \end{equation*}
\end{corollary}

\code{}{}{myhill_nerode_equiv_equal_suffix_head}
\codeblock{}{}{myhill_nerode_distinct_not_equal_suffix_head}
\codeblock{}{}{myhill_nerode_equivP_head}

\begin{definition}
    \label{f_min}
    Let $w \in \Sigma^*$. We define
    %Then $\mathit{f_{min}}$ is defined such that 
    \begin{equation*}
        \mathit{f_{min}}(w) := \{ x \; | \; x \in X, \; {f_0}(w) \cong x \}.
    \end{equation*}
    Note that the domain of $\mathit{f_{min}}$ is finite (since $X$ is finite) and contains no empty sets (due to reflexivity of $\cong$).
\end{definition}

\begin{lemma}
    \label{f_min_surjective}
    $\mathit{f_{min}}$ is surjective.
\end{lemma}
\begin{proof}
    Let $s \in dom(\mathit{f_{min}})$. 
    There exists $x \in X$ such that $x \in s$ since $s \neq \emptyset$. 
    We have ${f_0}(x) = {f_0}(\crep{x}$) and therefore $x \cong_{f_0} \crep{x}$ by reflexivity of $\cong$.
    Thus, $\crep{x}$ is a representative of $s$ since $\mathit{f_{min}}(x) = \mathit{f_{min}}(\crep{x}) = s$.
\end{proof}

\begin{lemma}
    \label{f_minP}
    For all $u,v \in \Sigma^*$ we we have 
    \begin{equation*}
        \mathit{f_{min}}(u) = \mathit{f_{min}}(v) \iff u \cong_{f_0} v.
    \end{equation*}
\end{lemma}

\begin{proof}
    ``$\Rightarrow$''
    We have $\mathit{f_{min}}(u) = \mathit{f_{min}}(v)$ and thus $u \cong_{f_0} v$.

    ``$\Leftarrow$''
    We have $u \cong_{f_0} v$. 
    Let $x \in X$.
    It suffices to show that ${f_0}(u) \cong x$ if and only if ${f_0}(v) \cong x$.
    This follows from symmetry and transitivity of $\cong$.
\end{proof}

\begin{corollary}
    \label{f_min_correct}
    $\mathit{f_{min}}$ is equivalent to the Nerode relation, i.e. $\mathit{f_{min}}$ is surjective and for all $u,v \in \Sigma^*$ we have
    \begin{equation*}
        \mathit{f_{min}}(u) = \mathit{f_{min}}(v) \iff u \doteq_L v.
    \end{equation*}
\end{corollary}

\begin{proof}
    We have proven surjectivity in lemma \ref{f_min_surjective}. 
    By lemma \ref{f_minP} we have $\mathit{f_{min}}(u) = \mathit{f_{min}}(v)$ if and only if $u \cong_{f_0} v$.
    By corollary \ref{equivP} we have $u \cong_{f_0} v$ if and only if $u \doteq_L v$.
    Thus, $\mathit{f_{min}}(u) = \mathit{f_{min}}(v)$ if and only if $u \doteq_L v$.
\end{proof}

\paragraph{}
The formalization of $\mathit{f_{min}}$ is slightly more involved than the mathematical construction. 
We first need to define the finite type of $\mathit{f_{min}}$'s domain, 
which we do by enumerating all possible values of $\mathit{f_{min}}$.

\code{}{}{myhill_nerode_equiv_repr}
\codeblock{}{}{myhill_nerode_X_min}
\codeblock{}{}{myhill_nerode_f_min}

\paragraph{}
We then prove lemmas \ref{f_min_surjective}, \ref{f_minP} and theorem \ref{f_min_correct} which are consequential and straight-forward.

\code{}{}{myhill_nerode_f_min_surjective_head}
\codeblock{}{}{myhill_nerode_f_minP_head}
\codeblock{}{}{myhill_nerode_f_min_correct_head}
\codeblock{}{}{myhill_nerode_f_min_fin}

\paragraph{}
We can now state the result of this chapter. 

\begin{corollary}
    The Nerode relation is of finite index.
\end{corollary}
\begin{proof}
    This follows directly from $\mathit{f_{min}}$.
\end{proof}

\code{}{}{myhill_nerode_weak_nerode_to_nerode}




\section{Finite Automata and Myhill-Nerode}


\subsection{Finite Automata to Myhill-Nerode}


\subsection{Myhill-Nerode to Finite Automata}

\paragraph{}


