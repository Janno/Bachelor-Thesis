\chapter{Decidable Languages}
\label{chap:lang}


We give basic definitions for languages, operators on languages and, finally, regular languages.
We provide the corresponding formalizations from our development and prove their correctness.

\section{Definitions}

We closely follow the definitions from \cite{DBLP:books/daglib/0011126}.
An \textbf{alphabet} $\Sigma$ is a finite set of symbols. 
A \textbf{word} $w$ is a finite sequence of symbols chosen from some alphabet. 
We use $|w|$ to denote the \textbf{length} of a word $w$. $\varepsilon$ denotes the empty word.
Given two words $w_1 = a_1 \cdots a_n$ and $w_2 = b_1 \cdots b_m$, the \textbf{concatenation} of $w_1$ and $w_2$ is defined as $a_1 \cdots a_n b_1 \cdots b_m$ and denoted $w_1 \cdot w_2$ or just $w_1 w_2$.
A \textbf{language} is a set of words.
The \textbf{residual language} of a language $L$ with respect to symbol $a$ is the set of words $u$ such that $a v$ is in $L$. The residual is denoted $res_a(L)$.
We define $\Sigma^k$ to be the \textbf{set of words of length k}.
The \textbf{set of all words} over an alphabet $\Sigma$ is denoted $\Sigma^*$, i.e.,$\Sigma^* = \bigcup_{k \in \mathbb{N}} \Sigma^k$.
A language $L$ is \textbf{decidable} if and only if there exists a boolean predicate that decides membership in $L$.
We will only deal with decidable languages from here on.
Throughout the remaining document, we will assume a fixed alphabet $\Sigma$.

     
    We employ finite types to formalize alphabets. 
    In most definitions, alphabets will not be made explicit.
    However, the same name and type will be used throughout the entire development.
    Words are formalized as sequences over the alphabet.
    Decidable languages are represented by functions from $word$ to $bool$.
    \code{language_char}{}{misc_char}
    \codeblock{language_word}{}{misc_word}
    \codeblock{language_language}{}{misc_language}
    \codeblock{language_residual}{}{regexp_residual}

    \subsection{Operations on Languages}

    
    We will later introduce a subset of the decidable language that is based on the following operations. For every operator, we will prove the decidabality of the resulting language.

    

    The \textbf{concatenation} of two languages $L_1$ and $L_2$ is denoted $L_1 \cdot L_2$ and is defined as the set of words $w = w_1 w_2$ such that $w_1$ is in $L_1$ and $w_2$ is in $L_2$.
    The \textbf{Kleene closure} of a language $L$ is denoted $L^*$ and is defined as the set of words $w = w_1 \cdots w_k$ such that $w_1 \ldots w_k$ are in $L$. 
    Note that $\varepsilon \in L^*$ ($k=0$).
    We define the \textbf{complement} of a language $L$ as $L \backslash \Sigma^*$, which we write as $\neg L$.
    Furthermore, we make use of the standard set operations \textbf{union} and \textbf{intersection}.

    
    For our \coq\ development,
    we take Coquand and Siles's \cite{DBLP:conf/cpp/CoquandS11} implementation of these operators. 
    \lstinline{plus} and \lstinline{prod} refer to union and intersection, respectively. 
    Additionally, we also introduce the singleton languages (\lstinline{atom}), the empty language (\lstinline{void}) and the language containing only the empty word (\lstinline{eps}).

    \code{lang_conc}{}{regexp_conc}
    \codeblock{lang_star}{}{regexp_star}
    \codeblock{lang_compl}{}{regexp_compl}
    \codeblock{lang_plus}{}{regexp_plus}
    \codeblock{lang_prod}{}{regexp_prod}
    \codeblock{lang_atom}{}{regexp_atom}
    \codeblock{lang_void}{}{regexp_void}
    \codeblock{lang_eps}{}{regexp_eps}

    
    The definition of \lstinline{conc} is based on a characteristic property of the concatenation of two languages. The following lemma proves this property.

    \begin{lemma}
        \label{concP}
        Let $L_1, L_2, w = a_1 \cdots a_k$ be given. We have that
        \begin{equation*}
            w \in L_1 \cdot L_2  
            %\mbox{ if and only if } 
            \iff
            \exists n \in \mathbb{N}. 0 < n \le k \wedge a_1 \cdots a_{n-1} \in L_1 \wedge a_n \cdots a_k \in L_2.
        \end{equation*}
    \end{lemma}
    \begin{proof}
        ``$\Rightarrow$'' 
        From $w \in L_1 \cdot L_2$ we have $w_1, w_2$ such that $w = w_1 w_2 \wedge w_1 \in L_1 \wedge w_2 \in L_2$.
        We choose $n := |w_1|+1$. We then have that $a_1 \cdots a_{n-1} = a_1 \cdots a_{|w_1|} = w_1$ and $w_1 \in L_1$ by assumption.
        Similarly, $a_{n} \cdots a_k = $$ a_{|w_1| + 1} \cdots a_k = w_2$ and $w_2 \in L_2$ by assumption.
        \\
        ``$\Leftarrow$'' 
        We choose $w_1 := a_1 \cdots a_{n-1}$ and $w_2 := a_n \cdots a_k$. By assumption we have that $w = w_1 w_2$.
        We also have that $a_1 \cdots a_{n-1} \in L_1$ and $a_{n} \cdots a_k \in L_2$. 
        It follows that $w_1 \in L_1$ and $w_2 \in L_2$.
    \end{proof}
    
    \code{}{}{regexp_concP_head}

    
    The implementation of \lstinline{star} makes use of a property of the Kleene closure, 
    which is that any nonempty word in $L^*$ can be seen as the concatenation of a nonempty word in $L$ and a (possibly empty) word in $L^*$.
    This property allows us to implement star as a structurally recursive predicate.
    The following lemma proves the correctness of this property.

    \begin{lemma}
        \label{starP}
        Let $L, w = a_1 \cdots a_k$ be given. We have that
        \begin{equation*}
            w \in L^*   
            %\mbox{ if and only if } 
            \iff
                \left\{
                    \begin{array}{ll}
                         a_2 \cdots a_k \in res_{a_1}(L) \cdot L^*, & \mbox{if } |w| > 0; \\
                         w = \varepsilon, & \mbox{otherwise. }
                    \end{array}
                \right.
        \end{equation*}
    \end{lemma}
    \begin{proof}
        ``$\Rightarrow$'' 
        We do a case distinction on $|w| = 0$.
        \begin{enumerate}
            \item
                $|w| = 0$. It follows that $w = \varepsilon$.
            \item
                $|W| \neq 0$, i.e. $|w| > 0$.
                From $w \in L^*$ we have $w = w_1 \cdots w_l$ such that $w_1 \cdots w_l$ are in $L$.
                There exists a minimal $n$ such that $|w_n| > 0$ and for all $m < n$, $|w_m| = 0$.
                Let $w_n = b_1 \cdots b_p$. We have that $b_2 \cdots a_p \in res_{b_1}(L)$.
                Furthermore, we have that $w_{n+1} \cdots w_l \in L^*$.
                We also have $a_1 = b_1$ and $w = a_1 \cdots a_k = w_n \cdots w_l$. 
                Therefore, we have $a_2 \cdots a_k \in res_{a_1}(L) \cdot L^*$.
        \end{enumerate}
        ``$\Leftarrow$'' 
        We do a case distinction on the disjunction.
        \begin{enumerate}
            \item
                $w = \varepsilon$. Then $w \in L^*$ by definition.
            \item
                $a_2 \cdots a_k \in res_{a_1}(L) \cdot L^*$. 
                By Lemma \ref{concP} we have $n$ such that $a_2 \cdots a_{n-1} \in res_{a_1}(L)$ and $a_n \cdots a_k \in L^*$.
                By definition of $res$, we have $a_1 \cdots a_{n-1} \in L$.
                Furthermore, we also have $a_n \cdots a_k = w_1 \cdots w_l$ such that $w_1 \ldots w_l$ are in $L$.
                We choose $w_0 := a_1 \cdots a_{n-1}$.
                It follows that $w = w_0 w_1 \cdots w_l$ with $w_0, w_1, \cdots w_l$ in $L$.
                Therefore, $w \in L^*$.
        \end{enumerate}
    \end{proof}

    
    The formalization of Lemma \ref{starP} connects the formalization of \lstinline{star} to the mathematical definition. The propositional formula given here appears slightly more restrictive than our mathemitcal definition as it requires all words from $L$ to be nonempty. 
    Mathematically, however, this is no restriction. 
    
    %The advantage of this formalization is can be seen in later proofs, where induction over the length of a word $w \in L^*$ is necessary. 
    %In such places, we will use Lemma \ref{starP} to get $w_1 \ldots w_k \in L$ such that $w = w_1 \cdots w_k$ 
    %and proceed by induction over the list of words $w_1 \ldots w_k$. 
    %Since every word $w_i$ has a length of at least one, we shorten $w$ by at least one character. 
    %Thus, we can use the inductive hypothesis of the induction over $|w|$ to prove the claim.

    \code{regexp_starP}{Formalization of Lemma \ref{starP}}{regexp_starP_head}

    \begin{theorem}{}
        \label{DecLangClosed}
        The decidable languages are closed under concatenation, Kleene star, union, intersection and complement. 
    \end{theorem}
    \begin{proof}
        We have already give algorithms for all operators. 
        It remains to show that they are correct.
        For concatenation and the Kleene star, we have shown in Lemma \ref{concP} and Lemma \ref{starP} that the formalizations are equivalent to the mathematical definitions. 
        The remaining operators (union, intersection, complement) can be applied directly to the result of the languages' boolean decision functions.
    \end{proof}

    


    \section{Regular Languages}


    \begin{definition}{}
        \label{REG}
        The set of regular languages $REG$ is defined to be exactly those languages generated by the following inductive definition:
        
        \begin{mathpar}
            \inferrule{ }{\emptyset \in REG} \and
            \inferrule{ }{\{\varepsilon\} \in REG} \and
            \inferrule{a \in \Sigma}{\{a\} \in REG} \and
            \inferrule{L \in REG}{L^* \in REG} \and
            \inferrule{L_1 \in REG \\ L_2 \in REG}{L_1 \cup L_2 \in REG} \and
            \inferrule{L_1 \in REG \\ L_2 \in REG}{L_1 \cdot L_2 \in REG}
        \end{mathpar}
    \end{definition}


    \subsection{Regular Expressions}

     
    Regular expressions mirror the definition of regular languages very closely. 
    \begin{definition}
        \label{extregex}
    We will consider \textbf{extended regular expressions} that include negation (\lstinline{Not}), intersection (\lstinline{And}) and a single-symbol wildcard (\lstinline{Dot}). 
    Therefore, we have the following syntax for regular expressions:
    \begin{equation*}
        r,s := \emptyset \; | \; \varepsilon \; | \; . \; | \; a \; | \; r^* \; | \; r + s \; | \; r \& s \; | \; r s \; | \; \neg r
    \end{equation*}
        

     
    %The semantics of these constructors are as follows:
    The language of an extended regular expression is defined as follows:
    \begin{align*}
        \lang{\emptyset} & = \emptyset
        & 
        \lang{r^*} & = \lang{r}^* \\
        \lang{\varepsilon} & = \{\varepsilon\}
        & 
        \lang{r + s} & = \lang{r} \cup \lang{s} \\
        \lang{.} & = \Sigma
        & 
        \lang{r \& s} & = \lang{r} \cap \lang{s} \\
        \lang{a} & = \{a\}
        &
        \lang{r s} & = \lang{r} \cdot \lang{s}
    \end{align*}
    \end{definition}

    \begin{definition}
        We say that two regular expressions $r$ and $s$ are equivalent if and only if
        \begin{equation*}
            \lang{r} = \lang{s}.
        \end{equation*}
    \end{definition}
    We will later show that equivalence of regular expressions is decidable.

    
    
    We take the implementation of regular expressions from Coquand and Siles's development (\cite{DBLP:conf/cpp/CoquandS11}), which is also based on \ssreflect\ and comes with helpful infrastructure for our proofs.
    The semantics defined in Definition \ref{extregex} can be given as a boolean function.
    %They also give a decision function the language associated with a regular expression.
    

    \code{regexp}{Regular Expressions}{regexp_regular_expression}
    \codeblock{regexp_mem_reg}{}{regexp_mem_reg}

    
    We will later prove that extended regular expressions are equivalent to the inductive definition of regular languages in \ref{REG}.
    In order to do that, we introduce a predicate on regular expressions that distinguishes \textbf{standard regular expressions} 
    from \textbf{extended regular expressions} (as introduced above).
    The grammar of standard regular expression is as follows:
    \begin{equation*}
        r,s := \emptyset \; | \; \varepsilon \; | \; a \; | \; r^* \; | \; r + s \; | \; r s \; 
    \end{equation*}

    Note that standard regular expressions correspond directly to Definition \ref{REG}.

    %\code{re_standard}{}{re_standard_standard}

    \todo{Connect standard regexp to reg. languages}

