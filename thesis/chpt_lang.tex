\chapter{Decidable Languages}

\section{Definition}
\paragraph{}
We closely follow the definitions from \cite{DBLP:books/daglib/0011126}.
An \textbf{alphabet} $\Sigma$ is a finite set of symbols. 
A \textbf{word} $w$ is a finite sequence of symbols chosen from some alphabet. 
We use $|w|$ to denote the \textbf{length} of a word $w$. $\varepsilon$ denotes the empty word.
Given two words $w_1 = a_1 a_2 \cdots a_n$ and $w_2 = b_1 b_2 \cdots b_m$, the \textbf{concatenation} of $w_1$ and $w_2$ is defined as $a_1 a_2 \cdots a_n b_1 b_2 \cdots b_m$ and denoted $w_1 \cdot w_2$ or just $w_1 w_2$.
A \textbf{language} is a set of words.
The \textbf{residual language} of a language $L$ with respect to symbol $a$ is the set of words $u$ such that $a v$ is in $L$. The residual is denoted $res_a(L)$.
We define $\Sigma^k$ to be the \textbf{set of words of length k}.
The \textbf{set of all words} over an alphabet $\Sigma$ is denoted $\Sigma^*$, i.e.,$\Sigma^* = \bigcup_{k \in \mathbb{N}} \Sigma^k$.
A language $L$ is \textbf{decidable} if and only if there exists a boolean predicate that decides membership in $L$.
We will only deal with decidable languages from here on.
Throughout the remaining document, we will assume a fixed alphabet $\Sigma$.

    \paragraph{} 
    We employ finite types to formalize alphabets. 
    In the most definitions, alphabets will not be made explicit.
    However, the same name and type will be used throughout the entire development.
    Words are formalized as sequences over the alphabet.
    Decidable languages are represented by functions from $word$ to $bool$.
    \code{language_char}{}{misc_char}
    \codeblock{language_word}{}{misc_word}
    \codeblock{language_language}{}{misc_language}
    \codeblock{language_residual}{}{regexp_residual}

    \subsection{Operation on languages}

    \paragraph{}

    The \textbf{concatenation} of two languages $L_1$ and $L_2$ is denoted $L_1 \cdot L_2$ and is defined as the set of words $w = w_1 w_2$ such that $w_1$ is in $L_1$ and $w_2$ is in $L_2$.
    The \textbf{Kleene Star} (also called Kleene closure) of a language $L$ is denoted $L^*$ and is defined as the set of words $w = w_1 w_2 \cdots w_k$ such that $w_1, w_2, \cdots w_k$ are in $L$. $\varepsilon$ is contained in $L^*$ ($k=0$).
    We define the \textbf{complement} of a language $L$ as $L \backslash \Sigma^*$, which we denote $\neg L$.
    Furthermore, we make use of the standard set operations \textbf{union} and \textbf{intersection}.

    \paragraph{}
    We take Coquand and Siles's \cite{DBLP:conf/cpp/CoquandS11} implementation of these operators. 
    \lstinline{plus} and \lstinline{prod} refer to union and intersection, respectively. 
    Additionally, we also introduce the singleton languages (\lstinline{atom}), the empty language (\lstinline{void}) and the language containing only the empty word (\lstinline{eps}).

    \code{lang_conc}{}{regexp_conc}
    \codeblock{lang_star}{}{regexp_star}
    \codeblock{lang_compl}{}{regexp_compl}
    \codeblock{lang_plus}{}{regexp_plus}
    \codeblock{lang_prod}{}{regexp_prod}
    \codeblock{lang_atom}{}{regexp_atom}
    \codeblock{lang_void}{}{regexp_void}
    \codeblock{lang_eps}{}{regexp_eps}

    %\paragraph{}
    %The implementation of \lstinline{star} 

    \begin{lemma}
        \label{concP}
        Let $L_1, L_2, w = a_1 a_2 \cdots a_k$ be given. We have that
        \begin{equation*}
            w \in L_1 \cdot L_2  
            %\mbox{ if and only if } 
            \iff
            \exists n \in \mathbb{N}. 0 < n \le k \wedge a_1 \cdots a_{n-1} \in L_1 \wedge a_n \cdots a_k \in L_2.
        \end{equation*}
    \end{lemma}
    \begin{proof}
        ``$\Rightarrow$'' 
        From $w \in L_1 \cdot L_2$ we have $w_1, w_2$ such that $w = w_1 w_2 \wedge w_1 \in L_1 \wedge w_2 \in L_2$.
        We choose $n := |w_1|+1$. We then have that $a_1 \cdots a_{n-1} = a_1 \cdots a_{|w_1|} = w_1$ and $w_1 \in L_1$ by assumption.
        Similarly, $a_{n} \cdots a_k = $$ a_{|w_1| + 1} \cdots a_k = w_2$ and $w_2 \in L_2$ by assumption.
        \\
        ``$\Leftarrow$'' 
        We choose $w_1 := a_1 \cdots a_{n-1}$ and $w_2 := a_n \cdots a_k$. By assumption we have that $w = w_1 w_2$.
        We also have that $a_1 \cdots a_{n-1} \in L_1$ and $a_{n} \cdots a_k \in L_2$. 
        It follows that $w_1 \in L_1$ and $w_2 \in L_2$.
    \end{proof}
    

    \code{regexp_concP}{Formalization of lemma \ref{concP}}{regexp_concP_head}

    \begin{lemma}
        \label{starP}
        Let $L, w = a_1 a_2 \cdots a_k$ be given. We have that
        \begin{equation*}
            w \in L^*   
            %\mbox{ if and only if } 
            \iff
            a_2 \cdots a_k \in res_{a_1}(L) \cdot L^* \vee w = \varepsilon.
        \end{equation*}
    \end{lemma}
    \begin{proof}
        ``$\Rightarrow$'' 
        We do a case distinction on $|w| = 0$.
        \begin{enumerate}
            \item
                $|w| = 0$. It follows that $w = \varepsilon$.
            \item
                $|W| \neq 0$, i.e. $|w| > 0$.
                From $w \in L^*$ we have $w = w_1 w_2 \cdots w_l$ such that $w_1, w_2 \cdots w_l$ are in $L$.
                There exists a minimal $n$ such that $|w_n| > 0$ and for all $m < n$, $|w_m| = 0$.
                Let $w_n = b_1 b_2 \cdots b_p$. We have that $b_2 \cdots a_p \in res_{b_1}(L)$.
                Furthermore, we have that $w_{n+1} \cdots w_l \in L^*$.
                We also have $a_1 = b_1$ and $w = a_1 a_2 \cdots a_k = w_n \cdots w_l$. 
                Therefore, we have $a_2 \cdots a_k \in res_{a_1}(L) \cdot L^*$.
        \end{enumerate}
        ``$\Leftarrow$'' 
        We do a case distinction on the disjunction.
        \begin{enumerate}
            \item
                $w = \varepsilon$. Then $w \in L^*$ by definition.
            \item
                $a_2 \cdots a_k \in res_{a_1}(L) \cdot L^*$. 
                By lemma \ref{concP} we have $n$ such that $a_2 \cdots a_{n-1} \in res_{a_1}(L)$ and $a_n \cdots a_k \in L^*$.
                By definition of $res$, we have $a_1 \cdots a_{n-1} \in L$.
                Furthermore, we also have $a_n \cdots a_k = w_1 w_2 \cdots w_l$ such that $w_1, w_2 \cdots w_l$ are in $L$.
                We choose $w_0 := a_1 \cdots a_{n-1}$.
                It follows that $w = w_0 w_1 \cdots w_l$ with $w_0, w_1, \cdots w_l$ in $L$.
                Therefore, $w \in L^*$.
        \end{enumerate}
    \end{proof}

    \code{regexp_starP}{Formalization of lemma \ref{starP}}{regexp_starP_head}

    \begin{theorem}{}
        \label{DecLangClosed}
        The decidable languages are closed under concatenation, Kleene star, union, intersection and complement. 
    \end{theorem}
    \begin{proof}
        We have already given algorithms for every operator. 
        It remains to show that they are correct.
        For concatenation and the Kleene star, we have shown in lemma \ref{concP} and \ref{starP} that the formalization is equivalent to the formal definition. 
        The remaining operators are applied directly to the decision functions. 
    \end{proof}

    \paragraph{}


    \section{Regular Languages}


    \begin{definition}{}
        \label{REG}
        The set of regular languages $REG$ is defined to be exactly those languages generated by the following inductive definition:
        
        \begin{mathpar}
            \inferrule{ }{\emptyset \in REG} \and
            \inferrule{ }{\{\varepsilon\} \in REG} \and
            \inferrule{a \in \Sigma}{\{a\} \in REG} \and
            \inferrule{L \in REG}{L^* \in REG} \and
            \inferrule{L_1, L_2 \in REG}{L_1 \cup L_2 \in REG} \and
            \inferrule{L_1, L_2 \in REG}{L_1 \cdot L_2 \in REG}
        \end{mathpar}
    \end{definition}


    \subsection{Regular Expressions}

    \paragraph{} 
    Regular expressions mirror the definition of regular languages very closely. 
    We will consider \textbf{extended regular expressions} that include negation ($Not$), intersection ($And$) and a single-symbol wildcard ($Dot$). 
    Therefore, we have the following syntax for regular expressions:
    \begin{equation*}
        r,s := \emptyset \; | \; \varepsilon \; | \; . \; | \; a \; | \; r^* \; | \; r + s \; | \; r \& s \; | \; r s \; | \; \neg r
    \end{equation*}

    \paragraph{} 
    The semantics of these constructors are as follows:
    \begin{enumerate}
        \item $\lang{\emptyset} = \emptyset$
        \item $\lang{\varepsilon} = \{\varepsilon\}$
        \item $\lang{.} = \Sigma$
        \item $\lang{a} = \{a\}$
        \item $\lang{r^*} = \lang{r}^*$
        \item $\lang{r + s} = \lang{r} \cup \lang{s}$
        \item $\lang{r \& s} = \lang{r} \cap \lang{s}$
        \item $\lang{r s} = \lang{r} \cdot \lang{s}$
    \end{enumerate}

    We take the implementation from Coquand and Siles's development (\cite{DBLP:conf/cpp/CoquandS11}), which is also based on \ssreflect\ and comes with helpful infrastructure for our proofs.

    \code{regexp}{Regular Expressions}{regexp_regular_expression}
    \codeblock{regexp_mem_reg}{}{regexp_mem_reg}

    \paragraph{}
    We will later prove that this definition is equivalent to the inductive definition of regular languages in \ref{REG}.
    In order to do that, we introduce a predicate on regular expressions that distinguishes \textbf{standard regular expressions} 
    from \textbf{extended regular expressions} (as introduced above).
    The grammar of standard regular expression is as follows:
    \begin{equation*}
        r,s := \emptyset \; | \; \varepsilon \; | \; a \; | \; r^* \; | \; r + s \; | \; r s \; 
    \end{equation*}


    \code{re_standard}{}{re_standard_standard}

    \todo{Connect standard regexp to reg. languages}


    \subsection{Deciding Language Membership}
    \paragraph{}
    We make use of \textbf{derivatives of regular expressions} (\cite{DBLP:journals/jacm/Brzozowski64}) to decide if a word $w \in \Sigma^*$ is contained in the language $\lang{r}$ of the regular expression $r$. 
    Derivatives are themselves regular expressions and are computed with respect to a single input character. 
    In order to define derivatives, we first define a related concept.

    %\begin{definition}{}
    %    \label{residual}
    %    The residual of a regular expression w.r.t to a character $a$ is the set of words $w$ such that $a \cdot w \in L$.
    %    \code{regexp_resiual}{}{regexp_residual}
    %\end{definition}

    \begin{definition}{}
        \label{derivative}
        The derivative $der \,  a \, r$ of $r$ w.r.t. to $a$ is defined such that 
        \begin{equation*}       \forall w \in \Sigma^*. \, w \, \in \, \lang{der \, a \, r} \Leftrightarrow w \, \in \, residual \, a \, \lang{r}. \end{equation*}
        \end{definition}

        \paragraph{}
        A suitable implementation is provided by Coquand and Siles.

        \code{regexp_der}{Derivatives of Regular Expressions}{regexp_der}


        \begin{theorem}
            \label{der_correct}
            For all $r$, $w$ and $a$, we have that $w \in der \, a \, r$ if and only if $w \in residual \, a $.
        \end{theorem}

        \begin{proof}
            We prove the claim by induction over $r$. Two cases are non-trivial: 
%                        \begin{enumerate}
%                            \item
%                                $der \, a \, (Conc \, r_1 \, r_2)$: 
%                                We have to check if $r_1$ accepts the empty word. 
%                                If it does not, the derivative is $Conc \, (der \, a \, r_1) \, r_2$.
%                                If it does, we also have to consider the case that the caracter $a$ is a prefix of a word in $\lang{r_2}$.
%                            \item
%                                $Star \, r_1$: 
%                                We have that $w \in \lang{r_1^*}$ if and only if 
%                                \todo{Fix one def. of *}
%                                \begin{equation*}
%                                    \, \exists n. \, \exists v_1 .. v_n. \, w \, = \, v_1 \cdot .. \cdot v_n 
%                                    \wedge \forall i. \, 1 \leq i \leq n 
%                                    \Rightarrow
%                                    \, v_i \in \lang{r_1}.
%                                \end{equation*}
%                        \end{enumerate}
            \todo{Proof}
        \end{proof}

        \paragraph{} 
        Given the defining property of derivatives, we can easily see that a generalization of $der$ to words suffices to decide language membership. We only need to check if the derivative w.r.t. to a given word accepts the empty word.

        \code{regexp_mem_der}{}{regexp_mem_der}

        \begin{theorem}
            \label{mem_der_correct} 
            The language of a regular expression $r$ is decidable, i.e.
            \begin{equation*}     w \in \lang{r} \Leftrightarrow \varepsilon \in \lang{mem\_der \, r \, w}.      \end{equation*}
            \end{theorem}
            \begin{proof}
                \todo{Proof}.
            \end{proof}



