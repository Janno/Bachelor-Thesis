\documentclass{beamer}

%  \setcounter{errorcontextlines}{999}

\usetheme{Madrid}
\usecolortheme{dove}
\useinnertheme{circles}



\usepackage{cite} 
\usepackage{natbib} %citep and citet

\usepackage[hidelinks]{hyperref} 

\usepackage{listings}

\newcommand{\coq}{\textsc{Coq}}
\newcommand{\ssreflect}{\textsc{SSReflect}}

\newcommand{\lang}[1]{\mathcal{L}(#1)}
\newcommand{\acc}[2]{\mathcal{L}_{#1}(#2)}
\newcommand{\crep}[1]{cr(#1)}

\newcommand{\last}[1]{ {#1}_{|{#1}|-1} }
\newcommand{\belast}[1]{ {#1}_1 \ldots {#1}_{|{#1}|-2} } 

\newcommand{\code}[3]{
    \lstinputlisting[label=#1,caption=#2,language=coq]{../docs/definitions/#3}
}

\newcommand{\codeblock}[3]{
    \vspace{-1em}
    \code{#1}{#2}{#3}
}


\newcounter{dummy} \numberwithin{dummy}{section}


\newtheorem{theorem}[dummy]{Theorem}
\newtheorem{lemma}[dummy]{Lemma}
\newtheorem{definition}[dummy]{Definition}
\newtheorem{corollary}[dummy]{Corollary}

\newcommand{\atoz}[0]{a b c d e f g h i j k l m n o p q r s t u v w x y z}
\newcommand{\AtoZ}[0]{A B C D E F G H I J K L M N O P Q R S T U V W X Y Z}

\newcommand*\circled[1]{\tikz[baseline=(char.base)]{
    \node[shape=circle,draw,inner sep=2pt] (char) {\vphantom{\atoz\AtoZ}#1};}}

\newcommand*\tcircled[1]{\tikz[baseline=(char.base)]{
    \node[shape=circle,draw,inner sep=0.5pt] (char) {\vphantom{\atoz\AtoZ}#1};}}



\newcommand{\dotcup}{\ensuremath{\mathaccent\cdot\cup}}







%\usepackage{cite} 

\usepackage{biblatex}
\bibliography{bib}{}
\renewcommand{\footnotesize}{\tiny}


%\def\newblock{\hskip .11em plus .33em minus .07em}
%\renewcommand{\bibsection}{\subsubsection*{\bibname } }


\beamertemplatenavigationsymbolsempty



\begin{document}
\title[Constr. Formalization of Reg. Languages]{Constructive Formalization of Regular Languages}  
\author[Jan-Oliver Kaiser]{Jan-Oliver Kaiser \\{\small Advisors: Christian Doczkal, Gert Smolka }\\{\small Supervisor: Gert Smolka}}
\institute{ }

\date{\today} 


\begin{frame}
    \titlepage
\end{frame}

\begin{frame}
    \tableofcontents
\end{frame}




\section{Recap}
\subsection*{Regular Expressions}
\begin{frame}[fragile]
    \frametitle{Regular Expressions}
    \framesubtitle{Definition}
    %\textbf{Definitions:} \\

    \begin{itemize}
        \item 
            We use extended Regular Expressions (RE) over an alphabet $\Sigma$:
            \begin{equation*}    
                r; s ::= \emptyset \; | \; \varepsilon \; | \;  a \; | \; rs \; | \;  r \;  + \; s \; | \; r \; \& \; s \; |\; r^* \; | \; \neg r
            \end{equation*}\vspace{-2em}%
            \begin{align*}%
                \lang{\emptyset} & = \emptyset
                & 
                \lang{r^*} & = \lang{r}^* \\
                \lang{\varepsilon} & = \{\varepsilon\}
                & 
                \lang{r + s} & = \lang{r} \cup \lang{s} \\
                \lang{.} & = \Sigma
                & 
                \lang{r \& s} & = \lang{r} \cap \lang{s} \\
                \lang{a} & = \{a\}
                &
                \lang{r s} & = \lang{r} \cdot \lang{s}
            \end{align*}
        \item Implementation taken from Coquand and Siles\footfullcite{DBLP:conf/cpp/CoquandS11}.
            \textbf{This saved us a lot of time}.
        \item $\approx 150$ lines of code including an implementation of regular languages and lots of lemmas.
    \end{itemize}

\end{frame}

\subsection*{Finite Automata}
\begin{frame}
    \frametitle{Finite Automata}
    \framesubtitle{Definition}
    \begin{itemize}
        \item 
            Our Finite Automata (FA) over an alphabet $\Sigma$ consists of
            \begin{enumerate}
                \item a set of states $Q$,
                \item a starting state $q_0 \in Q$,
                \item a set of final states $F \subseteq Q$,
                \item a transition relation $\delta \in Q \times \Sigma \times Q$.
            \end{enumerate}
        \item Two types, one for non-deterministic FA ($\delta$ may be non-functional), one for deterministic FA ($\delta$ is functional).
        \item For our deterministic FA, $\delta$ is also total and, thus, a function. This helped, but maybe not by much.
        \item Our definition is very close to the textbook definition.
        \item $\approx 120$ lines of code.
    \end{itemize}
\end{frame}

\subsection*{RE => FA}
\begin{frame}
    \frametitle{$\mbox{RE} \Rightarrow \mbox{FA}$}
    \begin{itemize}
        \item Structure of proof given by inductive definition of RE.
        \item Construct FA for every RE constructor.
        \item Sounds simple enough ..
        \item .. $\approx 700$ lines of code.
        \item This is a candidate for improvement.
        \item $\approx 100$ lines of code due to the fact that we use extended regular expressions.
    \end{itemize}
\end{frame}

\subsection*{Last Talk}
\begin{frame}
    \frametitle{$\mbox{FA} \Rightarrow \mbox{RE}$}
    %\framesubtitle{Last Talk}
    \begin{itemize}
        \item We use the ``Transitive Closure method'', Kleene's original proof.
        \item This method recursively builds a regular expression $R^X_{x,y}$ that recognizes words whose runs starting in $x$ only pass through states in $X$ and end in $y$.
        \item The previous version constructed $R^k_{x,y}$ which translates to $R^{\{z | \#(z) < k\}}_{x,y}$ where $\#$ is an ordering on $Q$.
        \item Instead of \textbf{nat}, we now recurse on the size of a \textbf{finite subset} of $Q$.\footfullcite{DBLP:books/daglib/0088160}
        \item This also avoids cumbersome conversion from \textbf{nat} to \ssreflect's ordinals and, finally, to states.
    \end{itemize}
\end{frame}

\begin{frame}
    \frametitle{$\mbox{FA} \Rightarrow \mbox{RE}$}
    \begin{itemize}
        \item After some restructuring: $\approx 550$ lines of code.
        \item Previous version: $\approx 800$ lines of code, much harder to read.
    \end{itemize}
    \begin{figure}
    \begin{minipage}[t]{0.5\textwidth}

        \lstinputlisting{L_split_old.v}
    \end{minipage}% \begin{minipage}{1in}
    \begin{minipage}[t]{0.5\textwidth}
        \lstinputlisting{L_split.v}
    \end{minipage}
        \caption{Previous and current version of the same lemma}
    \end{figure}
\end{frame}

\section{Myhill-Nerode Theorem}
\begin{frame}
    \frametitle{Myhill-Nerode Theorem}
    \begin{itemize}
        \item It turns out that there are two different concepts: Myhill relations and the Nerode relation.
        \item We also consider a related characterization: weak Nerode relations.
    \end{itemize}
\end{frame}

\section{Goal}
\subsection*{Current Goal}
\begin{frame}

    \textbf{Goal:}

    Find simple proofs for the decidability of regular expression equivalence and the Myhill-Nerode theorem.\\

    \textbf{Roadmap:}

    \begin{enumerate}
        \item RE $\Rightarrow$ FA (\textbf{DONE})
        \item Emptiness test on FA (\textbf{Easy})
        \item RE equivalence (\textbf{Follows from 1 and 2})
        \item FA $\Rightarrow$ RE (\textbf{Work in progress})
        \item Myhill-Nerode
    \end{enumerate}

\end{frame}

\section*{Finite automata to regular expressions}
\subsection*{Difficulty}
\begin{frame}

    \large{\textbf{Finite automata to regular expressions}}

    \begin{itemize}
        \item
            Converting REs to FAs is straight-forward and there is really only one algorithm (with slight variations): \\
            We mirror the constructors of RE in operations on FAs.

            \pause

        \item
            Converting FAs to REs is complicated and there are at least three algorithms found in textbooks.
    \end{itemize}

    \pause

    {\centering 
        \textbf{Why is that?}

    }

\end{frame}

\begin{frame}
    \textbf{My intuition:} 
    \begin{itemize}
        \item
            Converting REs to FAs is done by structural recursion on a \textbf{tree}. The result is a \textbf{flat structure}.

            \pause

        \item
            Converting FAs to REs \textbf{can not be done} by structural recursion.
            There is \textbf{no recursive structure} in FAs. \\
            But somehow we need to construct a \textbf{tree} of REs.
    \end{itemize}



\end{frame}

\subsection*{Overview}
\begin{frame}

    \textbf{Three methods (+ variations):}

    \begin{enumerate}
        \item State Removal (Du, Ko \cite{DuKo}, simplified in Linz \cite{DBLP:books/daglib/0019552})
        \item Brzozowski Algebraic Method (Brzozowski \cite{DBLP:journals/jacm/Brzozowski64})
        \item Transitive Closure (Kleene \cite{KleeneNets})
    \end{enumerate}

\end{frame}


\section{State Removal}
\subsection*{Approach}
\begin{frame}
    \textbf{State Removal}    

    \textbf{Given}: NFA $A$ = $(\Sigma, Q, q_0, F, \delta)$.

    \textbf{New concept}: Automata that have transitions labeled by RE.

    \textbf{Idea}: Remove states until there are two or less states remaining. Update the remaining states' transitions by incorporating the "lost" paths.


\end{frame}

\subsection*{Algorithm}
\begin{frame}
    \textbf{Remove $q$ from}\\

    \begin{figure}
        \begin{tikzpicture}[shorten >=1pt,node distance=2cm,>=stealth,thick]
            \node[state] (1) {$q_i$};
            \node[state] (2) [right of=1] {$q$};
            \node[state] (3) [right of=2] {$q_k$};
            \draw [->] (1) to[bend left] node[auto] {$a$} (2);
            \draw [->] (2) to[bend left] node[auto] {$b$} (3);
            \draw [->] (2) to[loop above] node[auto] {$e$} (2);
            \draw [->] (3) to[bend left] node[auto] {$c$} (2);
            \draw [->] (2) to[bend left] node[auto] {$d$} (1);
        \end{tikzpicture}
        \\
    \end{figure}

    \pause

    \textbf{to get}\\

    \begin{figure}
        \begin{tikzpicture}[shorten >=1pt,node distance=2cm,>=stealth,thick]
            \node[state] (1) {$q_i$};
            \node[state] (3) [right of=2] {$q_k$};
            \draw [->] (1) to[bend left] node[auto] {$ae^*b$} (3);
            \draw [->] (1) to[loop above] node[auto] {$ae^*d$} (1);
            \draw [->] (3) to[bend left] node[auto] {$ce^*d$} (1);
            \draw [->] (3) to[loop above] node[auto] {$ce^*b$} (3);
        \end{tikzpicture}
        \\
    \end{figure}

\end{frame}

\begin{frame}
    \textbf{Repeat until $A$ is of this form:}

    \begin{figure}
        \begin{tikzpicture}[shorten >=1pt,node distance=2cm,>=stealth,thick]
            \node[state] (1) {$q_1$};
            \node[state] (3) [right of=2, accepting] {$q_f$};
            \draw [->] (1) to[bend left] node[auto] {$r_2$} (3);
            \draw [->] (1) to[loop above] node[auto] {$r_1$} (1);
            \draw [->] (3) to[bend left] node[auto] {$r_3$} (1);
            \draw [->] (3) to[loop above] node[auto] {$r_4$} (3);
        \end{tikzpicture}
        \\
        $\Rightarrow \mathcal{L}(A) = \mathcal{L}(r_1^* r_2 (r_4 + r_3 r_1^* r_2)^*)$
    \end{figure}


\end{frame}

\subsection*{Caveats}
\begin{frame}
    \textbf{Caveats} \\
    \begin{itemize}
        \item
            It looks like we only need to update two edges. \\
            \pause
            In reality, there can be $|Q|-1$ states connected to $q$.
            \pause

            \begin{figure}
                \begin{tikzpicture}[shorten >=1pt,node distance=1.5cm,>=stealth,thick,minimum size=1cm]
                    \node[state] (q11) {};
                    \node[state] (q12) [right of = q11] {};
                    \node[state] (q13) [right of = q12] {};
                    \node[state] (q21) [below of = q11] {};
                    \node[state] (q22) [right of = q21] {$q$};
                    \node[state] (q23) [right of = q22] {};
                    \node[state] (q31) [below of = q21] {};
                    \node[state] (q32) [right of = q31] {};
                    \node[state] (q33) [right of = q32] {};

                    \draw [->] (q11) to[bend left=12.5] node[auto] {} (q22);
                    \draw [->] (q22) to[bend left=12.5] node[auto] {} (q11);

                    \draw [->] (q12) to[bend left=12.5] node[auto] {} (q22);
                    \draw [->] (q22) to[bend left=12.5] node[auto] {} (q12);

                    \draw [->] (q13) to[bend left=12.5] node[auto] {} (q22);
                    \draw [->] (q22) to[bend left=12.5] node[auto] {} (q13);

                    \draw [->] (q21) to[bend left=12.5] node[auto] {} (q22);
                    \draw [->] (q22) to[bend left=12.5] node[auto] {} (q21);

                    \draw [->] (q23) to[bend left=12.5] node[auto] {} (q22);
                    \draw [->] (q22) to[bend left=12.5] node[auto] {} (q23);

                    \draw [->] (q31) to[bend left=12.5] node[auto] {} (q22);
                    \draw [->] (q22) to[bend left=12.5] node[auto] {} (q31);

                    \draw [->] (q32) to[bend left=12.5] node[auto] {} (q22);
                    \draw [->] (q22) to[bend left=12.5] node[auto] {} (q32);

                    \draw [->] (q33) to[bend left=12.5] node[auto] {} (q22);
                    \draw [->] (q22) to[bend left=12.5] node[auto] {} (q33);


                \end{tikzpicture}
            \end{figure}


    \end{itemize}
\end{frame}

\begin{frame}
    \textbf{Caveats} \\
    \begin{itemize}
        \item
            It looks like we only need to update two edges. \\
            In reality, there can be $|Q|-1$ states connected to $q$.
        \item
            What about final states?\\
            \pause
            \begin{enumerate}
                \item
                    Introduce a new final state without any outgoing edges.
                \item
                    Introduce $\varepsilon$ transitions from all other final states to the final new state.
                \item
                    Make all other states non-final.
                \item
                    Never remove the new final state.
            \end{enumerate}
    \end{itemize}
\end{frame}

\subsection*{Properties}
\begin{frame}
    \textbf{Formalization:} \\
    \begin{itemize}
        \item
            Requires a new kind of finite automaton that has RE transitions.
        \item
            Lots of details to consider.
        \item
            Induction on the number of states.
    \end{itemize}
\end{frame}

\section{Brzozowski Algebraic Method}
\subsection*{Approach}
\begin{frame}
    \textbf{Brzozowski Algebraic Method}

    \textbf{Given}: FA $A$ = $(\Sigma, Q, q_0, F, \delta)$.\\
    \textbf{Idea}: Retrieve a RE for FA by solving a system of equations determined by $\delta$.

\end{frame}

\subsection*{Algorithm}
\begin{frame}
    \textbf{Construct system of equations}:

    \begin{equation*}
        \begin{array}{lcll} 
            r_0 & = & 
            \displaystyle\sum\limits_{
                \substack{a \in \Sigma \\ 0 \leq i < |Q|} 
            }
            \{ \, a \, r_i \, |  \, (q_0, a, q_i) \in \delta \} 
            &
            (+ \, \varepsilon \mbox{ if } r_0 \in F)
            \\ 
            \vdots &  = & \vdots \\
            r_{|Q|-1} & = & 
            \displaystyle\sum\limits_{
                \substack{a \in \Sigma \\ 0 \leq i < |Q|} 
            }
            \{ \, a \, r_i \, |  \, (q_{|Q|-1}, a, q_i) \in \delta \} 
            &
            (+ \, \varepsilon \mbox{ if } r_{|Q|-1} \in F)
            \\ 
        \end{array}
    \end{equation*}
    $\, \Rightarrow \mathcal{L}(A) = \mathcal{L}(r_0)$
\end{frame}

\begin{frame}
    Solve the system by substitution and \textbf{Arden's Lemma} which states that for all regular languages X, Y and Z the equation
    \begin{equation}
        \begin{array}{lcl}
            X = YX + Z
        \end{array}
    \end{equation}

    has the unique solution

    \begin{equation}
        \begin{array}{lcl}
            X = Y^*Z
        \end{array}
    \end{equation}
\end{frame}

\subsection*{Properties}
\begin{frame}
    \textbf{Formalization:} \\
    \begin{itemize}
        \item
            Requires a formalization of these equations and operations on them. 
        \item
            We would need to prove Arden's Lemma.
        \item
            We would also need to prove that Arden's Lemma (and substitution) is enough to solve these systems of equations.
    \end{itemize} 
\end{frame}

\section{Transitive Closure}
\subsection*{Approach}
\begin{frame}

    \large{\textbf{Transitive Closure}} \\

    \textbf{Given}: FA $A$ = $(\Sigma, Q, q_0, F, \delta)$.

    \textbf{Idea}: Construct regexps $r_f$ for all final states $f \in F$ s.t. \\$r_f$ matches all words which A accepts with final state $f$. \\

    \begin{equation*}
        \, \Rightarrow \mathcal{L}(A) = \mathcal{L}(\sum\limits_{f \in F} r_f)
    \end{equation*}

    \vspace{5 mm}

    \textbf{How do we construct $r_f$?} 

\end{frame}

\subsection*{Algorithm}
\begin{frame}

    We generalize the idea of $r_f$ to $R^k_{i j}$ which matches all words which lead from state $i$ to $j$ while passing only through states with index smaller than $k$.

    \begin{enumerate}
        \item 
            Merge multiple edges between states to one unified edge.
        \item
            Construct regexp $R^k_{i j}$ recursively:

            \begin{description}

                \item[$R^0_{i j}$]
                    $ := \begin{cases} 
                        r & \mbox{if } i \neq j \wedge i \mbox{ has edge } r \mbox{ to j}  \\
                        \varepsilon + r & \mbox{if } i = j \wedge i \mbox{ has edge } r \mbox{ to j}  \\
                        \emptyset & \mbox{otherwise}
                    \end{cases}
                    $ 

                \item[$R^k_{i j}$]
                    $ := R^{k-1}_{i k} (R^{k-1}_{k k})^* R^{k-1}_{k j} + R^{k-1}_{i j}$

            \end{description}

    \end{enumerate}

    \begin{equation*} 
        \Rightarrow \mathcal{L}(A) = \mathcal{L}(\sum_{f \in F} r_f) = \mathcal{L}(\sum_{f \in F} R^{|Q|}_{q_0 f}) 
    \end{equation*}

\end{frame}

\subsection*{Properties}
\begin{frame}
    \textbf{Formalization:} \\
    \begin{itemize}
        \item
            Easier than the other methods.\\
        \item
            The recursive definition translates quite well.\\
        \item
            The details are quite challenging.
    \end{itemize} 

    \pause

    {\centering 
        \textbf{This appears to be the simplest to formalize.}

    }
\end{frame}

\section{Our Approach}
\subsection*{Definitions}
\begin{frame}
    \textbf{Our Approach:} \\

    There are different ways of formalizing $R^k_{i j}$ itself, especially its parameters. Most practical so far: \\
    $k$ is of type nat, $i$ and $j$ are ordinals from $[0..|Q|-1]$.\\
    \pause
    This gives us easy recursion and matching on $k$.\\
    \pause
    \textbf{But} we have to use $min \, k \, (|Q|-1)$ to map k to the corresponding ordinal (and then to a state).

\end{frame}

\begin{frame}
    To get a FA counterpart to $R^k_{i j}$, we introduce $A^k_{i j}$ s.t.
    \begin{equation*}
        \mathcal{L}(A^k_{i j}) = \mathcal{L}(R^k_{i j}).
    \end{equation*}
    $A^k_{i j}$ is similar to A. It has one additional state, which has all incoming edges of state $j$ and no outgoing edges. It only leaves states that are $i$ or  $<k$. It only enters states $<k$ or the new state.
    \\
    We can then show that  
    \begin{equation*}
        \mathcal{L}(\bigcup_{f \in F} A^{|Q|}_{q_0 f}) = \mathcal{L}(A).
    \end{equation*}
\end{frame}

\subsection*{Lemmas}
\begin{frame}
    \textbf{Lemmas:}


    \begin{enumerate}
        \item The language of an automaton is the union of the words with runs on $A$ that end in any final state, i.e. 
            $\mathcal{L}(A) = \bigcup_{f \in F} \mathcal{L}(A_f) \mbox {, where $A_f$ accepts only in $f$}.$ (Not difficult)
        \item
            A path in $A^{k+1}_{i j}$ that is not in $A^{k}_{i j}$ can be decomposed into:\\
            \begin{itemize}
                \item a sub path from index 0 to the first occurrence of k
                \item a sub path from the first to the last occurrence of k
                \item the remaining sub path from the last occurrence of k to the end.
            \end{itemize}
            (Complex; requires new operations on lists and paths)
    \end{enumerate}

\end{frame}

\begin{frame}

    \begin{enumerate}

            \setcounter{enumi}{2}
        \item If a path on $A$ passes through a final state more than once, the corresponding word is in $\mathcal{L}(A)^*$ (Relies on the same operations we need for lemma 2)
        \item 
            $\mathcal{L}(A^k_{i j}) = \mathcal{L}(R^k_{i j})$ (By recursion on $k$, using lemmas 1, 2 and 3)
    \end{enumerate}
\end{frame}

\begin{frame}
    \begin{enumerate}
            \setcounter{enumi}{4}
        \item Any path on $A^k_{i j}$ that ends in the new accepting state can be mapped to an equivalent path that ends state j (and vice-versa). (Probably easy)
        \item Any path on $A^k_{i j}$ that does not end in the new accepting state is also a path on $A$. (Follows from lemma 5 and definition of $A^k_{i j}$)
        \item $\mathcal{L}(A^{|Q|}_{q_0 f}) = \mathcal{L}(A_f)$ (Follows from lemma 6)
    \end{enumerate}

\end{frame}

\subsection*{Theorem}
\begin{frame}
    \textbf{Theorem:}
    \begin{equation*}
        \mathcal{L}(\sum_{f \in F} R^{|Q|}_{q_0 f}) = \mathcal{L}(A).
    \end{equation*}

    \textbf{Proof:}\\
    \begin{itemize}
        \item By lemma 1:
            $\mathcal{L}(\sum_{f \in F} R^{|Q|}_{q_0 f}) = \bigcup_{f \in F} \mathcal{L}(A_f).$
        \item By lemma 7:
            $\mathcal{L}(\sum_{f \in F} R^{|Q|}_{q_0 f}) = \bigcup_{f \in F} \mathcal{L}(A^{|Q|_{q_0 f}}).$
        \item By lemma 4: 
            $\mathcal{L}(\sum_{f \in F} R^{|Q|}_{q_0 f}) = \bigcup_{f \in F} \mathcal{L}(R^{|Q|_{q_0 f}}).$
        \item Holds by definition of $+$.
    \end{itemize}

\end{frame}

\section{Summary}
\begin{frame}
    \textbf{Summary:}
    \begin{itemize}
        \item
            Creating recursive from non-recursive structures is difficult.
        \item 
            Existing algorithms to construct RE from FA differ vastly in how easily and elegantly we can formalize them.
        \item
            We benefit from thinking of the $R^k_{i j}$ invariant in terms of existing infrastructure ($A^k_{i j}$ is an ordinary NFA).
    \end{itemize}
\end{frame}

\begin{frame}
    \begin{center}
        \huge Thank you for your attention
    \end{center}
\end{frame}

%\begin{frame}[allowframebreaks]{Reference}
%    \bibliography{bib}{}
%    \bibliographystyle{plain}
%\end{frame}


\end{document}
