\documentclass{beamer}

%  \setcounter{errorcontextlines}{999}

\usetheme{Madrid}
\usecolortheme{dove}
\useinnertheme{circles}



\usepackage{cite} 
\usepackage{natbib} %citep and citet

\usepackage[hidelinks]{hyperref} 

\usepackage{listings}
%\usepackage{lstlangcoq}


\usepackage{color}
\usepackage{xcolor}

\usepackage{amsmath, amsthm, amssymb}


%% Coq listings
\lstset{% general command to set parameter(s)
    basicstyle=\small\sf, % print whole listing small%
        keywordstyle=\color{red}\bfseries, % underlined bold black keywords %magenta
        keywordstyle={[2]\color{green!60!black}\bfseries},
        identifierstyle=, % nothing happens
            commentstyle=\color{white}, % white comments
            stringstyle=\ttfamily, % typewriter type for strings+++
            showstringspaces=false, % no special string spaces
            columns=flexible,
        keepspaces=true,
        literate=
            %  {<->}{{\large $\leftrightarrow$}}{3}
    %  {->}{{\large $\rightarrow$}}{2}
    {⊨}{{$\models$}}{2}
    {⊆}{{$\subseteq$}}{2}  
    {∩}{{$\cap$}}{2}
    {∪}{{$\cup$}}{2}
    {∀}{{$\forall$}}{2}
    {⊳}{{$\rhd$}}{2}
    {~}{{${\sim}$}}{1}
    {Δ}{{$\mathrm{\Delta}$}}{1}
    %  {Box}{{$\Box$}}{1}
}
\lstdefinelanguage{coq}
{
    morekeywords={
        Definition,Variable,Record,Structure,Fixpoint,Function,Section,End,Inductive,CoInductive,
        Module,Axiom,Parameter,Lemma,Theorem,Let,Variables,Hypothesis,Corollary,Notation
    },
        otherkeywords={Module Type},
        morekeywords=[2]{
            match,with,end,if,then,else,is,as,return,let:,forall,exists,exists2,
            forallb,existsb,existss,alls},
        sensitive=true,
}
%\lstnewenvironment{coq}[1][]{\lstset{language=coq,#1}}{}
%\lstMakeShortInline[language=coq]!%

\lstset{language=coq}
%\usepackage{caption}

%\DeclareCaptionFont{white}{\color{white}}
%\DeclareCaptionFormat{listing}{\colorbox{gray}{\parbox{\textwidth}{#3}}}
%\captionsetup[lstlisting]{format=listing,labelfont=white,textfont=white}

\newcommand{\coq}{\textsc{Coq}}
\newcommand{\ssreflect}{\textsc{SSReflect}}

\newcommand{\lang}[1]{\mathcal{L}(#1)}
\newcommand{\acc}[2]{\mathcal{L}_#1(#2)}
\newcommand{\crep}[1]{cr(#1)}

\newcommand{\code}[3]{
    \lstinputlisting[label=#1,caption=#2,language=coq]{../docs/definitions/#3}
}

\newtheorem{theorem}{Theorem}[section]
\newtheorem{lemma}{Lemma}[section]
\newtheorem{definition}{Definition}[section]






%\usepackage{cite} 

\usepackage{biblatex}
\bibliography{bib}{}
\renewcommand{\footnotesize}{\tiny}


%\def\newblock{\hskip .11em plus .33em minus .07em}
%\renewcommand{\bibsection}{\subsubsection*{\bibname } }


\beamertemplatenavigationsymbolsempty



\begin{document}
\title[Constr. Formalization of Reg. Languages]{Constructive Formalization of Regular Languages}  
\author[Jan-Oliver Kaiser]{Jan-Oliver Kaiser \\{\small Advisors: Christian Doczkal, Gert Smolka }\\{\small Supervisor: Gert Smolka}}
\institute{ }

\date{\today} 


\begin{frame}
    \titlepage
\end{frame}

\begin{frame}
    \frametitle{Table of Contents}
    \tableofcontents
\end{frame}

\section{Motivation}
\begin{frame}
    \frametitle{Motivation}
    \begin{itemize}
        \item \textbf{Goal:} Build an extensive, elegant, constructive formalization of regular languages that includes 
            \begin{enumerate}
                \item regular expressions, 
                \item the decidability of equivalence of regular expressions,
                \item finite automata, 
                \item and the Myhill-Nerode theorem.
            \end{enumerate}
    \end{itemize}
\end{frame}



\section{Recap}
\subsection{Regular Expressions}
\begin{frame}[fragile]
    \frametitle{Regular Expressions}
    \framesubtitle{Definition}
    %\textbf{Definitions:} \\

    \begin{itemize}
        \item 
            We use extended Regular Expressions (RE) over an alphabet $\Sigma$:
            \begin{equation*}    
                r; s ::= \emptyset \; | \; \varepsilon \; | \;  a \; | \; rs \; | \;  r \;  + \; s \; | \; r \; \& \; s \; |\; r^* \; | \; \neg r
            \end{equation*}\vspace{-2em}%
            \begin{align*}%
                \lang{\emptyset} & = \emptyset
                & 
                \lang{r^*} & = \lang{r}^* \\
                \lang{\varepsilon} & = \{\varepsilon\}
                & 
                \lang{r + s} & = \lang{r} \cup \lang{s} \\
                \lang{.} & = \Sigma
                & 
                \lang{r \& s} & = \lang{r} \cap \lang{s} \\
                \lang{a} & = \{a\}
                &
                \lang{r s} & = \lang{r} \cdot \lang{s}
            \end{align*}
        \item Implementation taken from Coquand and Siles\footfullcite{DBLP:conf/cpp/CoquandS11}.
            \textbf{This saved us a lot of time}.
        \item $\approx 200$ lines of code including an implementation of regular languages and lots of useful lemmas.
    \end{itemize}

\end{frame}

\subsection{Finite Automata}
\begin{frame}
    \frametitle{Finite Automata}
    \framesubtitle{Definition}
    \begin{itemize}
        \item 
            Our Finite Automata (FA) over an alphabet $\Sigma$ consist of
            \begin{enumerate}
                \item a set of states $Q$,
                \item a starting state $s \in Q$,
                \item a set of final states $F \subseteq Q$,
                \item a transition relation $\delta \in Q \times \Sigma \times Q$.
            \end{enumerate}
        \item Two types, one for non-deterministic FA ($\delta$ may be non-functional), one for deterministic FA ($\delta$ is functional).
        \item For our deterministic FA, $\delta$ is also total and, thus, a function. This helped, but maybe not by much.
        \item Our definition is very close to the textbook definition.
        \item $\approx 120$ lines of code.
    \end{itemize}
\end{frame}

\subsection{Regular Expression to Finite Automaton}
\begin{frame}
    \frametitle{$\mbox{RE} \implies \mbox{FA}$}
    \begin{itemize}
        \item Structure of proof given by inductive definition of RE.
        \item \textbf{Plan:} Construct FA for every RE constructor.
        \item Sounds simple enough ..
        \item .. $\approx 800$ lines of code.
        \item $\approx 100$ lines of that needed for equivalence of DFA and NFA.
        \item Another $\approx 100$ lines of code for \textbf{extended} regular expressions.
        \item This is a candidate for improvement.
    \end{itemize}
\end{frame}

\section{Previous Talk}
\subsection{Finite Automaton to Regular Expression}
\begin{frame}
    \frametitle{$\mbox{FA} \implies \mbox{RE}$}
    %\framesubtitle{Last Talk}
    \begin{itemize}
        \item We use the ``Transitive Closure method'', Kleene's original proof\footfullcite{KleeneNets}.
        \item This method recursively builds a regular expression $R^X_{x,y}$ that recognizes words whose runs starting in $x$ only pass through states in $X$ and end in $y$.
        \item The previous version constructed $R^k_{x,y}$ which translates to $R^{\{z | \#(z) < k\}}_{x,y}$ where $\#$ is an ordering on $Q$.
        \item Instead of \textbf{nat}, we now recurse on the size of a \textbf{finite subset} of $Q$.\footfullcite{DBLP:books/daglib/0088160}
        \item This also avoids cumbersome conversion from nat to \ssreflect's ordinals and, finally, to states.
    \end{itemize}
\end{frame}

\begin{frame}
    \frametitle{$\mbox{FA} \implies \mbox{RE}$}
    \framesubtitle{Proof Outline}
    \begin{itemize}
        \item We introduce a decidable language $L^X_{x,y}$ that directly encodes the idea behind $R^{X}_{x,y}$ as a boolean predicate.
        \item The proof of $\mbox{FA} \implies \mbox{RE}$ consists of three steps: 
            \begin{enumerate}
                \item We show that $L^X_{x,y}$ respects the defining, recursive equation of $R^X_{x,y}$.
                \item We show that for all $w \in \Sigma^*$, $w \in L^X_{x,y} \iff w \in R^X_{x,y}$.
                \item We show that $\bigcup_{f \in F} L^Q_{s,f} = \lang{A}$ and thus $\sum_{f \in F} R^Q_{s,f} = \lang{A}$. 
            \end{enumerate}
    \end{itemize}
\end{frame}


\begin{frame}
    \frametitle{$\mbox{FA} \implies \mbox{RE}$}
    \begin{itemize}
        \item After some restructuring: $\approx 550$ lines of code, $\approx 150$ of which are general infrastructure.
        \item Previous version: $\approx 800$ lines of code, much harder to read.
    \end{itemize}
    \begin{figure}
    \begin{minipage}[t]{0.5\textwidth}

        \lstinputlisting{L_split_old.v}
    \end{minipage}% \begin{minipage}{1in}
    \begin{minipage}[t]{0.5\textwidth}
        \lstinputlisting{L_split.v}
    \end{minipage}
        \caption{Previous and current version of the same lemma}
    \end{figure}
\end{frame}

\section{Myhill-Nerode Theorem}
\begin{frame}
    \frametitle{Myhill-Nerode Theorem}
    \begin{itemize}
        \item It turns out that there are two different concepts: Myhill relations and the Nerode relation.
        \item We also consider a related characterization: weak Nerode relations.
        \item All these are equivalence relations which we require to be of finite index, i.e. to have a finite number of equivalence classes.
        \item However, \coq\ has no notion of quotient types.
    \end{itemize}
\end{frame}


\subsection{Formalization of Equivalence Classes}
\begin{frame}
    \frametitle{Myhill-Nerode Theorem}
    \framesubtitle{Equivalence Relations of Finite Index}
    \begin{itemize}
        \item We use functions of finite domain to represent equivalence relations of finite index. 
        \item Think of the domain as the set of equivalence classes.
        \item We also need to have a representative of every equivalence class. Thus, we require surjectivity.
            \coderaw{}{}{myhill_nerode_Fin_Eq_Cls}
        \item With constructive choice, we can then give a canonical representative of every every equivalence class.
            \coderaw{}{}{myhill_nerode_cr}

    \end{itemize}
\end{frame}

\subsection{Myhill Relations and Nerode Relations}
\begin{frame}
    \frametitle{Myhill-Nerode Theorem}
    \framesubtitle{Myhill relations, weak Nerode relations, Nerode relation}
    \begin{itemize}
        \item An equivalence relation $\equiv$ of finite index is a \textbf{Myhill relation}\footfullcite{Myhill1957FiniteAutomataand} for L iff
            \begin{enumerate}
                \item $\equiv$ is \textbf{right-congruent}, i.e. $\forall u,v \in \Sigma^*. \; \forall a \in \Sigma. \; u \equiv v \implies ua \equiv va$, 
                \item and $\equiv$ \textbf{refines L}, i.e. $\forall u,v \in \Sigma^*. \; u \equiv v \implies (u \in L \iff v \in L)$.
            \end{enumerate}
        \item An equivalence relation $\equiv$ of finite index is a \textbf{weak Nerode relation} for L iff 
            \begin{equation*}
               \forall u,v \in \Sigma^*. \; u \equiv v \implies \forall w \in \Sigma^*. \; (uw \in L \iff vw \in L).
            \end{equation*}
        \item The \textbf{Nerode relation}\footfullcite{Nerode1958} $\doteq_L$ for L is defined such that
            \begin{equation*}
               \forall u,v \in \Sigma^*. \; u \doteq_L v \iff \forall w \in \Sigma^*. \; (uw \in L \iff vw \in L).
            \end{equation*}
    \end{itemize}
\end{frame}

\begin{frame}
    \frametitle{Myhill-Nerode Theorem}
    \framesubtitle{Formalization of Myhill, weak Nerode and Nerode relation}
    \begin{itemize}
        \item For all three relations, we build a record that consists of an equivalence relation of finite index and proofs of the properties of the respective relation. 
        \item Example:
    \coderaw{}{}{myhill_nerode_Myhill_Rel}
    \end{itemize}
\end{frame}

\begin{frame}
    \frametitle{Myhill-Nerode Theorem}
    \begin{itemize}
        \item Our version of the Myhill-Nerode theorem states that the following are equivalent
            \begin{enumerate}
                \item language L is accepted by a DFA,
                \item we can construct a Myhill relation for L,
                \item we can construct a weak Nerode relation for L,
                \item the Nerode relation for L is of finite index.
            \end{enumerate}
        \item 
            $(1) \implies (2)$ is easy. (Map word $w$ to the last state of its run on the automaton.)
        \item 
            $(2) \implies (3)$ is a trivial inductive proof. 
        \item
            $(4) \implies (1)$ is also straight-forward. (Use equivalence classes as states.)
    \end{itemize}
\end{frame}

\subsection{Minimizing Equivalence Classes}
\begin{frame}
    \frametitle{Myhill-Nerode Theorem}
    \framesubtitle{$(3) \implies (4)$}
    \begin{itemize}
        \item A weak Nerode relation (given as a function $f$) has at least as many equivalence classes as the Nerode relation.
        \item Some of them are redundant w.r.t. to L, i.e. we may have equivalence classes s.t.
            \begin{equation*}
                \forall u v \in \Sigma^*. \; f u \neq f u \wedge \forall w \in \Sigma^*. \; (uw \in L \iff vw \in L).
            \end{equation*}
        \item Our goal is to merge these equivalence classes. 
        \item In fact, we construct an equivalence relation on these equivalence classes.
        \item Equivalence classes are contained in that relation iff they are equivalent w.r.t. to L.
    \end{itemize}
\end{frame}

\begin{frame}
    \frametitle{Myhill-Nerode Theorem}
    \framesubtitle{Finding equivalent equivalence classes}
    \begin{itemize}
        \item We use the table-filling algorithm\footfullcite{Huffman1954161} for minimizing DFA. 
        \item Our fixed-point algorithm that finds all equivalence classes distinguishable w.r.t. to L.
        \item The remaining equivalence classes are then equivalent w.r.t. to L. 
        \item Start: $\{ (x,y) \; | \; cr(x) \notin L \iff cr(y) \in L \}$.
        \item Step: if the previous result is $d$, the new result is $d \cup \{ (x,y) \; | \; \exists a. \; (f(cr(x)a), f(cr(y)a)) \in d \}$.
        \item Due to finiteness of the domain and monotonicity of the algorithm, it has a fixed point which we can compute.
    \end{itemize}
\end{frame}

\begin{frame}
    \frametitle{Myhill-Nerode Theorem}
    \framesubtitle{Proof Outline}
    \begin{itemize}
        \item We construct a function $\mathit{f_{min}}$ that maps every equivalence class to the set of equivalence classes equivalent to it w.r.t. L.
        \item We then show that $\mathit{f_{min}}$ is surjective and encodes an equivalence relation of finite index.
        \item Finally, we show that $\mathit{f_{min}}$ is equivalent to the Nerode relation.
    \end{itemize}
\end{frame}

\begin{frame}
    \frametitle{Myhill-Nerode Theorem}
    \framesubtitle{}
    \begin{itemize}
        \item The lemmas of this chapter are rather abstract, which makes for nice and short statements.
        \item The proofs also received more refinement than the other chapters. They are very concise.
        \item The entire chapter consists of $\approx 430$ lines of code.
    \end{itemize}
\end{frame}

\section{Summary}
\begin{frame}
    \frametitle{Summary}
    \framesubtitle{}
    \begin{itemize}
        \item All in all we have $\approx 2100$ lines of code.  %430+550+820+200
        \item Mostly very close to the mathematical definitions.
        \item Code produced in the beginning of the project might be improved quite a bit.
    \end{itemize}
\end{frame}



\begin{frame}
    \begin{center}
        \huge Thank you for your attention
    \end{center}
\end{frame}

%\begin{frame}[allowframebreaks]{Reference}
%    \bibliography{bib}{}
%    \bibliographystyle{plain}
%\end{frame}


\end{document}
